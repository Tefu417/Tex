\documentclass[fleqn]{jsarticle}

\usepackage{amsmath, amssymb}
\usepackage{setspace}
\usepackage{nccmath}
\usepackage{type1cm}

\title{基礎数理演習課題6}
\author{21716070 縫嶋慧深}

\begin{document}

	\maketitle

    \section*{1}
    次の行列の計算をして下さい。演算が定義されない場合はその旨を答えて下さい。\\
    \begin{description}
		\setlength{\itemsep}{0.5cm}
        \begin{spacing}{1.2}

            \item[(1)]
                $ \left(
                    \begin{array}{cc}
                        3 & -1 \\
                        -5 & 1 \\
                        5 & 7
                    \end{array}
                \right)
                +
                \left(
                    \begin{array}{cc}
                        2 & 3 \\
                        5 & 6 \\
                        -4 & -9
                    \end{array}
                \right)
                =
                \left(
                    \begin{array}{cc}
                        5 & 2 \\
                        0 & 7 \\
                        1 & -2
                    \end{array}
                \right) $ \\

            \item[(2)]
                $ \left(
                    \begin{array}{cc}
                        8 & 6 \\
                        7 & 5
                    \end{array}
                \right)
                -
                4 \left(
                    \begin{array}{cc}
                        3 & 1 \\
                        2 & 0
                    \end{array}
                \right)
                =
                \left(
                    \begin{array}{cc}
                        -4 & 2 \\
                        -1 & 5
                    \end{array}
                \right) $ \\

            \item[(3)]
                $ \left(
                    \begin{array}{ccc}
                        3 & -2 & -1 \\
                        5 & 6 & -4
                    \end{array}
                \right)
                \left(
                    \begin{array}{cc}
                        -1 & 5 \\
                        3 & -2 \\
                        -1 & -1
                    \end{array}
                \right)
                =
                \left(
                    \begin{array}{cc}
                        -8 & 20 \\
                        17 & 17
                    \end{array}
                \right) $ \\

            \item[(4)]
                $ \left(
                    \begin{array}{cc}
                        1 & 2 \\
                        0 & 1 \\
                        -1 & 0 \\
                        -2 & -1
                    \end{array}
                \right)
                \left(
                    \begin{array}{cc}
                        2 & 1 \\
                        -1 & 1
                    \end{array}
                \right)
                =
                \left(
                    \begin{array}{cc}
                        0 & 3 \\
                        -1 & 1 \\
                        -2 & -1 \\
                        -3 & -3
                    \end{array}
                \right) $ \\

            \item[(5)]
                $ \left(
                    \begin{array}{cc}
                        3 & 5
                    \end{array}
                \right)
                \left(
                    \begin{array}{ccc}
                        1 & 3 & 2 \\
                        1 & 0 & -2
                    \end{array}
                \right)
                =
                \left(
                    \begin{array}{ccc}
                        8 & 9 & -4
                    \end{array}
                \right) $ \\

            \item[(6)]
                $ \left(
                    \begin{array}{cc}
                        \sqrt{2} & 1 \\
                        2 & \sqrt{2}
                    \end{array}
                \right)
                \left(
                    \begin{array}{cc}
                        \sqrt{2} & 1 \\
                        -1 & \sqrt{2}
                    \end{array}
                \right)
                =
                \left(
                    \begin{array}{cc}
                        1 & 2\sqrt{2} \\
                        \sqrt{2} & 4
                    \end{array}
                \right) $ \\

        \end{spacing}
    \end{description}

    \section*{2}
    次の実3次元数ベクトルに対し、$(1)\sim(3)$を求めて下さい。\\
    $ \vec{a}
    =
    \left(
        \begin{array}{c}
            0 \\
            1 \\
            1
        \end{array}
    \right)
    ,
    \vec{b}
    =
    \left(
        \begin{array}{c}
            2 \\
            1 \\
            3
        \end{array}
    \right)
    ,
    \vec{c}
    =
    \left(
        \begin{array}{c}
            4 \\
            -1 \\
            2
        \end{array}
    \right) $
    また、$\vec{a}=\vec{OA},\vec{b}=\vec{OB},\vec{c}=\vec{OC}$とする。

    \begin{description}
		\setlength{\itemsep}{0.5cm}
        \begin{spacing}{1.2}

            \item[(1)]
                $ (\vec{a}+\vec{b})\times\vec{c} $\\
                $ (\vec{a}+\vec{b})\times\vec{c}
                =
                \left(
                    \begin{array}{c}
                        2 \\
                        2 \\
                        4
                    \end{array}
                \right)
                \times
                \left(
                    \begin{array}{c}
                        4 \\
                        -1 \\
                        2
                    \end{array}
                \right)
                =
                8 + (-2) + 8
                =
                14
                $

            \item[(2)]
                $ \vec{a},\vec{b},\vec{c} $のなす平行六面体の体積 \\
                $ V =
                (\vec{a} \times \vec{b}) \cdot \vec{c}
                =
                (
                    \left(
                        \begin{array}{c}
                            0 \\
                            1 \\
                            1
                        \end{array}
                    \right)
                    \times
                    \left(
                        \begin{array}{c}
                            2 \\
                            1 \\
                            3
                        \end{array}
                    \right)
                )
                \cdot
                \left(
                    \begin{array}{c}
                        4 \\
                        -1 \\
                        2
                    \end{array}
                \right)
                =
                \left(
                    \begin{array}{c}
                        2 \\
                        2 \\
                        -2
                    \end{array}
                \right)
                \cdot
                \left(
                    \begin{array}{c}
                        4 \\
                        -1 \\
                        2
                    \end{array}
                \right)
                = 8 + (-2) + (-4)
                = 2
                $

            \item[(3)]
                $ \triangle ABC $ の面積 \\
                $ S =
                \frac{1}{2} \sqrt{|\vec{AB}|^2|\vec{AC}|^2 - (\vec{AB} \cdot \vec{AC})^2} $ より、\\
                $ \vec{AB} =
                \left(
                    \begin{array}{c}
                        0 \\
                        1 \\
                        1
                    \end{array}
                \right)
                \times
                \left(
                    \begin{array}{c}
                        2 \\
                        1 \\
                        3
                    \end{array}
                \right)
                =
                \left(
                    \begin{array}{c}
                        0 \\
                        1 \\
                        3
                    \end{array}
                \right) $ \\
                $ |\vec{AB}|
                = \sqrt{0 + 1 + 9}
                = \sqrt{10} $ \\
                $ \vec{AC} =
                \left(
                    \begin{array}{c}
                        0 \\
                        1 \\
                        1
                    \end{array}
                \right)
                \times
                \left(
                    \begin{array}{c}
                        4 \\
                        -1 \\
                        2
                    \end{array}
                \right)
                =
                \left(
                    \begin{array}{c}
                        0 \\
                        -1 \\
                        2
                    \end{array}
                \right) $ \\
                $ |\vec{AC}|
                = \sqrt{0 + 1 + 4}
                = \sqrt{5} $ \\
                $ \vec{AB} \cdot \vec{AC} =
                \left(
                    \begin{array}{c}
                        0 \\
                        1 \\
                        3
                    \end{array}
                \right)
                \times
                \left(
                    \begin{array}{c}
                        0 \\
                        -1 \\
                        2
                    \end{array}
                \right)
                = 0 + (-1) + 6
                = 5 $ \\
                $ S = \frac{1}{2} \sqrt{10 \times 5 - 25}
                = \frac{1}{2} \times 5
                = \frac{5}{2} $

        \end{spacing}
    \end{description}

    \section*{3}
    次の正方行列の逆行列を求めて下さい。持たない場合はその判定もして下さい。\\\\

    \begin{description}
		\setlength{\itemsep}{0.5cm}
        \begin{spacing}{1.2}

            \item[(1)]
                $ A =
                \left(
                    \begin{array}{c}
                        -\frac{3}{4}
                    \end{array}
                \right) $ \\\\
                $ A^{-1} =
                \left(
                    \begin{array}{c}
                        -\frac{4}{3}
                    \end{array}
                \right) $

            \item[(2)]
                $ A =
                \left(
                    \begin{array}{cc}
                        7 & 9 \\
                        4 & 5
                    \end{array}
                \right) $ \\\\
                $ detA = 35 - 36 = -1 $ \\\\
                $ A^{-1} =
                -\left(
                    \begin{array}{cc}
                        5 & -9 \\
                        -4 & 7
                    \end{array}
                \right)
                =
                \left(
                    \begin{array}{cc}
                        -5 & 9 \\
                        4 & -7
                    \end{array}
                \right) $

            \item[(3)]
                $ A =
                \left(
                    \begin{array}{cc}
                        4 & -6 \\
                        -6 & 9
                    \end{array}
                \right) $ \\\\
                $ detA = 36 - 36 = 0 $ \\
                $ \therefore A $ は逆行列を持たない。

            \item[(4)]
                $ A =
                \left(
                    \begin{array}{cc}
                        \sqrt{6} & -2 \\
                        1 & \sqrt{6}
                    \end{array}
                \right) $ \\\\
                $ detA = 6 - (-2) = 8 $ \\\\
                $ A^{-1} =
                \frac{1}{8} \left(
                    \begin{array}{cc}
                        \sqrt{6} & 2 \\
                        -1 & \sqrt{6}
                    \end{array}
                \right) $

            \item[(5)]
                $ A =
                \left(
                    \begin{array}{ccc}
                        -1 & 1 & 2 \\
                        2 & 6 & 4 \\
                        5 & 4 & -1
                    \end{array}
                \right) $ \\
                $ detA = 6 + 20 + 16 - 60 - (-2) - (-16) = 0 $ \\
                $ \therefore A $ は逆行列を持たない。

            \item[(6)]
                $ A =
                \left(
                    \begin{array}{ccc}
                        1 & 2 & -2 \\
                        2 & 2 & -3 \\
                        -1 & -1 & 2
                    \end{array}
                \right) $ \\\\
                $ detA = 4 + 6 + 4 - 4 - 8 - 3 = -1 $ \\\\
                $ \left(
                    \begin{array}{c|c}
                        A & E_n
                    \end{array}
                \right)
                =
                \left(
                    \begin{array}{ccc|ccc}
                        1 & 2 & -2 & 1 & 0 & 0 \\
                        2 & 2 & -3 & 0 & 1 & 0 \\
                        -1 & -1 & 2 & 0 & 0 & 1
                    \end{array}
                \right)
                \longleftrightarrow
                \left(
                    \begin{array}{ccc|ccc}
                        1 & 2 & -2 & 1 & 0 & 0 \\
                        0 & -2 & 1 & -2 & 1 & 0 \\
                        0 & 1 & 0 & 1 & 0 & 1
                    \end{array}
                \right) \\\\
                \longleftrightarrow
                \left(
                    \begin{array}{ccc|ccc}
                        1 & 2 & -2 & 1 & 0 & 0 \\
                        0 & 1 & 0 & 1 & 0 & 1 \\
                        0 & -2 & 1 & -2 & 1 & 0
                    \end{array}
                \right)
                \longleftrightarrow
                \left(
                    \begin{array}{ccc|ccc}
                        1 & 0 & -2 & -1 & 0 & -2 \\
                        0 & 1 & 0 & 1 & 0 & 1 \\
                        0 & 0 & 1 & 0 & 1 & 2
                    \end{array}
                \right)
                \longleftrightarrow
                \left(
                    \begin{array}{ccc|ccc}
                        1 & 0 & 0 & -1 & 2 & 2 \\
                        0 & 1 & 0 & 1 & 0 & 1 \\
                        0 & 0 & 1 & 0 & 1 & 2
                    \end{array}
                \right) \\\\
                =
                \left(
                    \begin{array}{c|c}
                        E_n & B
                    \end{array}
                \right) $ \\\\
                $ \therefore
                A^{-1} =
                \left(
                    \begin{array}{ccc}
                        -1 & 2 & 2 \\
                        1 & 0 & 1 \\
                        0 & 1 & 2
                    \end{array}
                \right) $

            \item[(7)]
                $ A =
                \left(
                    \begin{array}{cccc}
                        -1 & 0 & 1 & 1 \\
                        0 & 0 & 1 & 1 \\
                        1 & 1 & 0 & 0 \\
                        1 & 1 & 0 & -1
                    \end{array}
                \right) $ \\\\
                $ \left(
                    \begin{array}{c|c}
                        A & E_n
                    \end{array}
                \right)
                =
                \left(
                    \begin{array}{cccc|cccc}
                        -1 & 0 & 1 & 1 & 1 & 0 & 0 & 0 \\
                        0 & 0 & 1 & 1 & 0 & 1 & 0 & 0 \\
                        1 & 1 & 0 & 0 & 0 & 0 & 1 & 0 \\
                        1 & 1 & 0 & -1 & 0 & 0 & 0 & 1
                    \end{array}
                \right)
                \longleftrightarrow
                \left(
                    \begin{array}{cccc|cccc}
                        1 & 0 & -1 & -1 & -1 & 0 & 0 & 0 \\
                        1 & 1 & 0 & 0 & 0 & 0 & 1 & 0 \\
                        0 & 0 & 1 & 1 & 0 & 1 & 0 & 0 \\
                        1 & 1 & 0 & -1 & 0 & 0 & 0 & 1
                    \end{array}
                \right) \\
                \longleftrightarrow
                \left(
                    \begin{array}{cccc|cccc}
                        1 & 0 & -1 & -1 & -1 & 0 & 0 & 0 \\
                        0 & 1 & 1 & 1 & 1 & 0 & 1 & 0 \\
                        0 & 0 & 1 & 1 & 0 & 1 & 0 & 0 \\
                        0 & 1 & 1 & 0 & 1 & 0 & 0 & 1
                    \end{array}
                \right)
                \longleftrightarrow
                \left(
                    \begin{array}{cccc|cccc}
                        1 & 0 & -1 & -1 & -1 & 0 & 0 & 0 \\
                        0 & 1 & 1 & 1 & 1 & 0 & 1 & 0 \\
                        0 & 0 & 1 & 1 & 0 & 1 & 0 & 0 \\
                        0 & 0 & 0 & -1 & 0 & 0 & -1 & 1
                    \end{array}
                \right) \\
                \longleftrightarrow
                \left(
                    \begin{array}{cccc|cccc}
                        1 & 0 & 0 & 0 & -1 & 1 & 0 & 0 \\
                        0 & 1 & 0 & 0 & 1 & -1 & 1 & 0 \\
                        0 & 0 & 1 & 1 & 0 & 1 & 0 & 0 \\
                        0 & 0 & 0 & -1 & 0 & 0 & -1 & 1
                    \end{array}
                \right)
                \longleftrightarrow
                \left(
                    \begin{array}{cccc|cccc}
                        1 & 0 & 0 & 0 & -1 & 1 & 0 & 0 \\
                        0 & 1 & 0 & 0 & 1 & -1 & 1 & 0 \\
                        0 & 0 & 1 & 1 & 0 & 1 & 0 & 0 \\
                        0 & 0 & 0 & 1 & 0 & 0 & 1 & -1
                    \end{array}
                \right) \\
                \longleftrightarrow
                \left(
                    \begin{array}{cccc|cccc}
                        1 & 0 & 0 & 0 & -1 & 1 & 0 & 0 \\
                        0 & 1 & 0 & 0 & 1 & -1 & 1 & 0 \\
                        0 & 0 & 1 & 0 & 0 & 1 & -1 & 1 \\
                        0 & 0 & 0 & 1 & 0 & 0 & 1 & -1
                    \end{array}
                \right)
                =
                \left(
                    \begin{array}{c|c}
                        E_n & B
                    \end{array}
                \right) $ \\\\
                $ \therefore
                A^{-1} =
                \left(
                    \begin{array}{cccc}
                        -1 & 1 & 0 & 0 \\
                        1 & -1 & 1 & 0 \\
                        0 & 1 & -1 & 1 \\
                        0 & 0 & 1 & -1
                    \end{array}
                \right) $

        \end{spacing}
    \end{description}

    \section*{4}
    次の連立一次方程式を解いて下さい。持たない場合はその判定もして下さい。
    ただし、$ l = 0 $

    \begin{description}
		\setlength{\itemsep}{0.5cm}
        \begin{spacing}{1.2}

            \item[(1)]
            $ \left\{
                \begin{array}{l}
                    x + y = -1 \\
                    3x + 4y = 4 \\
                    5x + 6y = 0
                \end{array}
            \right. $ \\\\
            拡大係数行列を簡約化する。\\\\
                $ \left(
                    \begin{array}{cc|c}
                        1 & 1 & -1 \\
                        3 & 4 & 4 \\
                        5 & 6 & 0
                    \end{array}
                \right)
                \longleftrightarrow
                \left(
                    \begin{array}{cc|c}
                        1 & 1 & -1 \\
                        0 & 1 & 7 \\
                        0 & 1 & 5
                    \end{array}
                \right)
                \longleftrightarrow
                \left(
                    \begin{array}{cc|c}
                        1 & 0 & -8 \\
                        0 & 1 & 7 \\
                        0 & 0 & -2
                    \end{array}
                \right) $ \\\\
                拡大係数行列の階数は$3$、係数行列の階数は$2$なので、両者は一致せず、解を持たない。

            \item[(2)]
                $ \left\{
                    \begin{array}{l}
                        x + y - z = 1 \\
                        -3x - 2y = -2 \\
                        6x - 2z = 7
                    \end{array}
                \right. $ \\\\
                拡大係数行列を簡約化する。\\\\
                $ \left(
                    \begin{array}{ccc|c}
                        1 & 1 & -1 & 1 \\
                        -3 & -2 & 0 & -2 \\
                        6 & 0 & -2 & 7
                    \end{array}
                \right)
                \longleftrightarrow
                \left(
                    \begin{array}{ccc|c}
                        1 & 1 & -1 & 1 \\
                        0 & 1 & -3 & 1 \\
                        0 & -6 & 4 & 1
                    \end{array}
                \right)
                \longleftrightarrow
                \left(
                    \begin{array}{ccc|c}
                        1 & 0 & 2 & 0 \\
                        0 & 1 & -3 & 1 \\
                        0 & 0 & -14 & 7
                    \end{array}
                \right)
                \longleftrightarrow
                \left(
                    \begin{array}{ccc|c}
                        1 & 0 & 2 & 0 \\
                        0 & 1 & -3 & 1 \\
                        0 & 0 & 1 & -\frac{1}{2}
                    \end{array}
                \right) \\
                \longleftrightarrow
                \left(
                    \begin{array}{ccc|c}
                        1 & 0 & 0 & 1 \\
                        0 & 1 & 0 & -\frac{1}{2} \\
                        0 & 0 & 1 & -\frac{1}{2}
                    \end{array}
                \right) $ \\\\
                連立方程式に戻すと、
                $ \left\{
                    \begin{array}{l}
                        x = 1 \\
                        y = -\frac{1}{2} \\
                        z = -\frac{1}{2}
                    \end{array}
                \right. $

            \item[(3)]
                $ \left\{
                    \begin{array}{l}
                        x + 2y + 3z + 4w = 3 \\
                        -x + z + 2w = -1 \\
                        3x - y - 5z - 9w = 2
                    \end{array}
                \right. $ \\\\
                拡大係数行列を簡約化する。\\\\
                $ \left(
                    \begin{array}{cccc|c}
                        1 & 2 & 3 & 4 & 3 \\
                        -1 & 0 & 1 & 2 & -1 \\
                        3 & -1 & -5 & -9 & 2
                    \end{array}
                \right)
                \longleftrightarrow
                \left(
                    \begin{array}{cccc|c}
                        1 & 2 & 3 & 4 & 3 \\
                        0 & 2 & 4 & 6 & 2 \\
                        0 & -7 & -14 & -21 & -7
                    \end{array}
                \right)
                \longleftrightarrow
                \left(
                    \begin{array}{cccc|c}
                        1 & 2 & 3 & 4 & 3 \\
                        0 & 1 & 2 & 3 & 1 \\
                        0 & 1 & 2 & 3 & 1
                    \end{array}
                \right) \\\\
                \longleftrightarrow
                \left(
                    \begin{array}{cccc|c}
                        1 & 0 & -1 & -2 & 1 \\
                        0 & 1 & 2 & 3 & 1 \\
                        0 & 0 & 0 & 0 & 0
                    \end{array}
                \right) $ \\\\
                連立方程式に戻すと、
                $ \left\{
                    \begin{array}{l}
                        x - z - 2w = 1 \\
                        y + 2z + 3w = 1
                    \end{array}
                \right. $ \\\\
                $z, w$ を $s, t \in \mathbb{R}$ とおくと、
                $ \left\{
                    \begin{array}{l}
                        x = s + 2w + 1 \\
                        y = 2s + 3w + 1 \\
                        z = s \\
                        w = t
                    \end{array}
                \right.
                (s, t \in \mathbb{R}) $

        \end{spacing}
    \end{description}

    \section*{[$+\alpha$]}
    問題$4(2)$の連立方程式を一般の整数$l$に対して解いて下さい。

    $ \left\{
        \begin{array}{l}
            x + y - z = 1 \\
            -3x - 2y + lz = -2 \\
            6x + ly - 2z = 7
        \end{array}
    \right. $ \\\\
    拡大係数行列を簡約化する。\\\\
        $ \left(
            \begin{array}{ccc|c}
                1 & 1 & -1 & 1 \\
                -3 & -2 & l & -2 \\
                6 & l & -2 & 7
            \end{array}
        \right)
        \longleftrightarrow
        \left(
            \begin{array}{ccc|c}
                1 & 1 & -1 & 1 \\
                0 & 1 & l-3 & 1 \\
                0 & l-6 & 4 & 1
            \end{array}
        \right)
        \longleftrightarrow
        \left(
            \begin{array}{ccc|c}
                1 & 0 & -l+2 & 0 \\
                0 & 1 & l-3 & 1 \\
                0 & 0 & -l^2+9l-14 & -l+5
            \end{array}
        \right) \\\\
        \longleftrightarrow
        \left(
            \begin{array}{ccc|c}
                1 & 0 & -l+2 & 0 \\
                0 & 1 & l-3 & 1 \\
                0 & 0 & l^2-9l+14 & l-5
            \end{array}
        \right) $ \\\\
        連立方程式に戻すと、
        $ \left\{
            \begin{array}{l}
                x + (-l + 2)z = 0 \\
                y + (l - 3)z = 1 \\
                (l^2-9l+14)z = l - 5
            \end{array}
        \right. $

        \begin{description}
            \setlength{\itemsep}{0.5cm}
            \begin{spacing}{1.2}

                \item[●$ l = 2, 7 $ の時]
                    $ \left\{
                        \begin{array}{l}
                            x + (-l + 2)z = 0 \\
                            y + (l - 3)z = 1 \\
                            0 = l - 5
                        \end{array}
                    \right. $ \\\\
                    拡大係数行列の階数は$3$、係数行列の階数は$2$となるので、両者は一致せず、解を持たない。

                \item[●$ l \neq 2, 7 $ の時]
                    $ \left\{
                        \begin{array}{l}
                            x - lz + 2z = 0 \\
                            y + lz - 3z = 1 \\
                            l^2z - 9lz + 14z = l - 5
                        \end{array}
                    \right. $ \\\\
                    $ x - lz + 2z = 0 $を$l$について解くと、\\
                    $ lz = x + 2z \\
                    l = \frac{x}{z} + 2 $

            \end{spacing}
        \end{description}



\end{document}