\documentclass[fleqn]{jsarticle}

\usepackage{amsmath, amssymb}
\usepackage{setspace}
\usepackage{nccmath}
\usepackage{type1cm}
\usepackage{multicol}

\title{基礎数理演習課題8}
\author{21716070 縫嶋慧深}

\begin{document}

	\maketitle

    \section*{0}
    次の値を求めて下さい。

    \begin{description}
        \setlength{\itemsep}{0.5cm}

        \begin{multicols}{2}

            \item[(1)]
                $ \sin^{-1}{\left(-\cfrac{\sqrt{3}}{2}\right)} $ \\
                $ \sin^{-1}{-\cfrac{\sqrt{3}}{2}} = -\cfrac{\pi}{3}\left(\in\left[-\frac{\pi}{2}, \frac{\pi}{2}\right]\right) $

            \item[(2)]
                $ \cos^{-1}{\left(-\cfrac{\sqrt{2}}{2}\right)} $ \\
                $ \cos^{-1}{\left(-\cfrac{\sqrt{2}}{2}\right)} = \cfrac{3\pi}{4}\left(\in\left[0, \pi\right]\right) $

        \end{multicols}

        \item[(5)]
            $ \tan^{-1}{\left(-\sqrt{3}\right)} $ \\
            $ \tan^{-1}{\left(-\sqrt{3}\right)} = -\cfrac{\pi}{3}\left(\in\left(-\frac{\pi}{2}, \frac{\pi}{2}\right)\right) $

    \end{description}


    \section*{1}
    次の関数の導関数を求めて下さい。

    \begin{description}
        \setlength{\itemsep}{0.5cm}

        \begin{multicols}{2}

            \item[(1)]
                $ f(x) = x^3 + x^2 - x - 1 $ \\
                $ f^{\prime}(x) = 3x^2 + 2x - 1 $


            \item[(2)]
                $ f(x) = x^{-2} + x^{\frac{1}{2}} $ \\
                $ f^{\prime}(x) = -2x^{-3} + \cfrac{x^{\left(-\frac{1}{2}\right)}}{2} $

        \end{multicols}

        \begin{multicols}{2}

            \item[(3)]
                $ f(x) = \cfrac{1}{x} - \sqrt[4]{x} $ \\
                $ f^{\prime}(x) = -\cfrac{1}{x^2} - \cfrac{1}{4x^{\frac{3}{4}}} $


            \item[(4)]
                $ f(x) = 3^x + log_3{x} $ \\
                $ f^{\prime}(x) = \cfrac{1}{xlog{3}} + 3^xlog{3} $

        \end{multicols}

        \begin{multicols}{2}

            \item[(5)]
                \begin{flalign*}
                    & \hspace*{-10mm} f(x) = \sqrt{x}\cos{x} \\
                    & \hspace*{-10mm} f^{\prime}(x) = \cfrac{1}{2\sqrt{x}} \cdot \cos{x} + \sqrt{x} \cdot (-\sin{x}) \\
                    & \hspace{-2mm} = \cfrac{\cos{x} - 2x\sin{x}}{2\sqrt{x}}
                \end{flalign*}


            \item[(6)]
                \begin{flalign*}
                    & \hspace*{-10mm} f(x) = x\sin^{-1}{x} \\
                    & \hspace*{-10mm} f^{\prime}(x) = \sin^{-1}{x} + x \cdot \cfrac{1}{\sqrt{1-x^2}} \\
                    & \hspace{-2mm} = \sin^{-1}{x} + \cfrac{x}{\sqrt{1-x^2}}
                \end{flalign*}

        \end{multicols}

        \begin{multicols}{2}

            \item[(7)]
            \hspace*{-10ex}\begin{flalign*}
                & \hspace*{-10mm} f(x) = \cfrac{-x}{x^2 - 2x + 1} \\
                & \hspace*{-10mm} f^{\prime}(x) = \cfrac{-(x^2 -  2x + 1) - (-x) \cdot (2x - 2)}{(x^2 - 2x + 1)^2} \\
                & \hspace*{-2mm} = \cfrac{x^2 - 1}{(x - 1)^4} \\
                & \hspace*{-2mm} = \cfrac{x + 1}{(x - 1)^3}
            \end{flalign*}

        \item[(8)]
            \begin{flalign*}
                & \hspace*{-10mm} f(x) = \cfrac{x}{log{x}} \\
                & \hspace*{-10mm} f^{\prime}(x) = \cfrac{log{x} - x \cdot \frac{1}{x}}{log^2{x}} \\
                & \hspace*{-2mm} = \cfrac{log{x} - 1}{log^2{x}}
            \end{flalign*}

        \end{multicols}

    \end{description}

    \section*{2}
    次の関数の導関数を求めて下さい。

    \begin{description}
        \setlength{\itemsep}{0.5cm}

        \begin{multicols}{2}

            \item[(1)]
                \begin{flalign*}
                    & \hspace*{-10mm} f(x) = (log{x})^2 \\
                    & \hspace*{-10mm} f^{\prime}(x) = 2log{x} \cdot \cfrac{1}{x} \\
                    & \hspace*{-2mm} = \cfrac{2log{x}}{x}
                \end{flalign*}

            \item[(2)]
                \begin{flalign*}
                    & \hspace*{-10mm} f(x) = e^{\tan^{-1}{x}} \\
                    & \hspace*{-10mm} f^{\prime}(x) =  e^{\tan^{-1}{x}} \cdot \cfrac{1}{x^2 + 1} \\
                    & \hspace*{-2mm} = \cfrac{e^{\tan^{-1}{x}}}{x^2 + 1}
                \end{flalign*}

        \end{multicols}

        \begin{multicols}{2}

            \item[(3)]
                \begin{flalign*}
                    & \hspace*{-10mm} f(x) = \cos^{-1}{2x} \\
                    & \hspace*{-10mm} f^{\prime}(x) = -\cfrac{1}{\sqrt{1 - (2x)^2}} \cdot 2 \\
                    & \hspace*{-2mm} = -\cfrac{2}{\sqrt{1 - 4x^2}}
                \end{flalign*}
                \linebreak
                \linebreak
                \linebreak
                \linebreak
                \linebreak
                \linebreak
                \linebreak

            \item[(4)]
                $ f(x) = x^{\sin{x}} \hspace{2ex} (x > 0) $ \\
                $ y = x^{\sin{x}} $ と置く。 \\
                両辺の自然対数を取ると、 \\
                $ \hspace{4ex} log{y} = log{x^{\sin{x}}} \\
                \hspace{8ex} = \sin{x} \cdot log{x} $ \\
                両辺を$x$で微分すると、 \\
                $ \hspace{4ex} \cfrac{y^{\prime}}{y} = \sin{x} \cdot \frac{1}{x} + \cos{x} \cdot log{x} \\
                \hspace{6ex} = \cfrac{\sin{x}}{x} + \cos{x}log{x} \\
                \hspace{4ex} y^{\prime} = y \cdot \left\{\cfrac{\sin{x}}{x} + \cos{x}log{x}\right\} \\
                \hspace{6ex} = x^{\sin{x}} \left\{\cfrac{\sin{x}}{x} + \cos{x}log{x}\right\} $

        \end{multicols}

    \end{description}

\end{document}