\documentclass[fleqn]{jsarticle}
\setlength{\headsep}{0truemm}
\addtolength{\footskip}{20mm}

\usepackage{amsmath, amssymb}
\usepackage{setspace}
\usepackage{nccmath}
\usepackage{type1cm}
\usepackage{multicol}

\makeatletter
\newcommand{\ccinc}{\ifx\@currsize\small
    \setlength{\unitlength}{1.1pt}
    \begin{picture}(10,10)(1,2)
    \put(1,2){\line(0,1){3}}
    \put(5,5){\oval(8,8)[lt]}
    \put(5,9){\vector(1,0){4}}
    \end{picture}\else
    \setlength{\unitlength}{1.2pt}
    \begin{picture}(10,10)(1,2)
    \put(1,2){\line(0,1){3}}
    \put(5,5){\oval(8,8)[lt]}
    \put(5,9){\vector(1,0){4}}
    \end{picture}\fi}
\newcommand{\cvinc}{\ifx\@currsize\small
    \setlength{\unitlength}{1.1pt}
    \begin{picture}(10,10)(2,1)
    \put(2,1){\line(1,0){3}}
    \put(5,5){\oval(8,8)[rb]}
    \put(9,5){\vector(0,1){4}}
    \end{picture}\else
    \setlength{\unitlength}{1.2pt}
    \begin{picture}(10,10)(2,1)
    \put(2,1){\line(1,0){3}}
    \put(5,5){\oval(8,8)[rb]}
    \put(9,5){\vector(0,1){4}}
    \end{picture}\fi}
\newcommand{\ccdec}{\ifx\@currsize\small
    \setlength{\unitlength}{1.1pt}
    \begin{picture}(10,10)(2,1)
    \put(2,9){\line(1,0){3}}
    \put(5,5){\oval(8,8)[rt]}
    \put(9,5){\vector(0,-1){4}}
    \end{picture}\else
    \setlength{\unitlength}{1.2pt}
    \begin{picture}(10,10)(2,1)
    \put(2,9){\line(1,0){3}}
    \put(5,5){\oval(8,8)[rt]}
    \put(9,5){\vector(0,-1){4}}
    \end{picture}\fi}
\newcommand{\cvdec}{\ifx\@currsize\small
    \setlength{\unitlength}{1.1pt}
    \begin{picture}(10,10)(1,0)
    \put(1,8){\line(0,-1){3}}
    \put(5,5){\oval(8,8)[lb]}
    \put(5,1){\vector(1,0){4}}
    \end{picture}\else
    \setlength{\unitlength}{1.2pt}
    \begin{picture}(10,10)(1,0)
    \put(1,8){\line(0,-1){3}}
    \put(5,5){\oval(8,8)[lb]}
    \put(5,1){\vector(1,0){4}}
    \end{picture}\fi}
\makeatother

\title{基礎数理演習課題13}
\author{21716070 縫嶋慧深}

\begin{document}

	\maketitle

    \section*{1}
    次の関数を微分して下さい。

    \begin{description}
		\setlength{\itemsep}{0.8cm}

            \begin{multicols}{2}

                \item[(1)]
                    $ f(x) = x^2 + 2^x + log_2{x} $
                    \begin{flalign*}
                        & \hspace*{-6mm} f(x) = x^2 + 2^x + \cfrac{log{x}}{log{2}} \\
                        & \hspace*{-6mm} f^{\prime}(x) = 2x + 2^xlog{2} + \cfrac{1}{xlog{2}}
                    \end{flalign*}

                \item[(2)]
                    $ f(x) = \tan{x}\sin^{-1}{x} $
                    \begin{flalign*}
                        & \hspace*{-6mm} f^{\prime}(x) = (\sin^{-1}{x})^{\prime} \cdot \tan{x} + \sin^{-1}{x} \cdot (\tan{x})^{\prime} \\
                        & \hspace*{2mm} = \cfrac{1}{\sqrt{1-x^2}} \cdot \tan{x} + \sin^{-1}{x} \cdot \cfrac{1}{\cos^2{x}} \\
                        & \hspace*{2mm} = \cfrac{\tan{x}}{\sqrt{1-x^2}} + \cfrac{\sin^{-1}{x}}{\cos^2{x}}
                    \end{flalign*}

            \end{multicols}

            \begin{multicols}{2}

                \item[(3)]
                    $ f(x) = \cfrac{1-\cos{x}}{x^2} $
                    \begin{flalign*}
                        & \hspace*{-6mm} f(x) = \sin{x}\cdot\cfrac{1}{x^2} + (1-\cos{x})\cdot\left(-\cfrac{2}{x^3}\right) \\
                        & \hspace*{-6mm} f^{\prime}(x) = \cfrac{x\sin{x}+2\cos{x}-2}{x^3}
                    \end{flalign*}

                \item[(4)]
                    $ f(x) = \sqrt{2e^x+1} $
                    \begin{flalign*}
                        & \hspace*{-6mm} f^{\prime}(x) = \cfrac{\left(2e^x+1\right)^{\prime}}{2\sqrt{2e^x+1}} \\
                        & \hspace*{2mm} = \cfrac{e^x}{\sqrt{2e^x+1}}
                    \end{flalign*}

            \end{multicols}

            \begin{multicols}{2}

                \item[(5)]
                    $ f(x) = \cfrac{1-\cos{x}}{x^2} $
                    \begin{flalign*}
                        & \hspace*{-6mm} f(x) = -\cfrac{(x^2)^{\prime}}{\sqrt{1-x^4}} \\
                        & \hspace*{2mm} = -\cfrac{2x}{\sqrt{1-x^4}}
                    \end{flalign*}

                \item[(6)]
                    $ f(x) = e^{2x}\sin{3x} $
                    \begin{flalign*}
                        & \hspace*{-6mm} f^{\prime}(x) = \left(e^{2x}\right)^{\prime}\cdot\sin{3x} + e^{2x}\cdot(\sin{3x})^{\prime} \\
                        & \hspace*{2mm} = e^{2x}\left(2\sin{3x} + 3\cos{3x}\right) \\
                    \end{flalign*}

            \end{multicols}

            \item[(7)]
                $ f(x) = x^{\cos{x}} \hspace*{4mm} (x>0) $
                \begin{flalign*}
                    & \hspace*{-6mm} 両辺の自然対数をとる \\
                    & \hspace*{-6mm} log{f(x)} = log{x^{\cos{x}}} \\
                    & \hspace*{-6mm} log{f(x)} = \cos{x}log{x} \\
                    & \hspace*{-6mm} 両辺を微分する \\
                    & \hspace*{-6mm} \cfrac{f^{\prime}(x)}{f(x)} = -\sin{x}log{x} + \cfrac{\cos{x}}{x} \\
                    & \hspace*{-6mm} f^{\prime}(x) = x^{\cos{x}}\left(-\sin{x}log{x} + \cfrac{\cos{x}}{x}\right) \\
                    & \hspace*{2mm} = x^{\cos{x}-1}\left(-x\sin{x}log{x} + \cos{x}\right)
                \end{flalign*}

    \end{description}

    \section*{2}
    次の値を求めて下さい。

    \begin{description}
        \setlength{\itemsep}{0.5cm}

        \begin{multicols}{2}

            \item[(1)]
                $ \sin^{-1}{\cfrac{\sqrt{3}}{2}} $ \\
                $ \sin^{-1}{\cfrac{\sqrt{3}}{2}} = \cfrac{\pi}{3} $

            \item[(2)]
                $ \cos^{-1}{\left(-\cfrac{1}{\sqrt{2}}\right)} $ \\
                $ \cos^{-1}{\left(-\cfrac{1}{\sqrt{2}}\right)} = \cfrac{3\pi}{4} $

        \end{multicols}

        \item[(3)]
            $ \tan^{-1}{\left(-\cfrac{1}{\sqrt{3}}\right)} $ \\
            $ \tan^{-1}{\left(-\cfrac{1}{\sqrt{3}}\right)} = -\cfrac{\pi}{6} $

    \end{description}

    \vspace{-5mm} \section*{3}
    次の不定積分を求めて下さい。(積分定数として$C, C_1, C_2, \cdots $を断らずに用いてよい)

    \begin{description}
        \setlength{\itemsep}{0.5cm}

        \begin{multicols}{2}

            \item[(1)]
                \begin{flalign*}
                    & \hspace*{-10mm} \int \sqrt{x}\left(x-1+\cfrac{1}{x}\right) \ dx \cdots (*) \\
                    & \hspace*{-10mm} u = \sqrt{x} とすると \\
                    & \hspace*{-10mm} (*) = 2\int (u^4-u^2+1) \ du \\
                    & \hspace*{-6mm} = \cfrac{2u^5}{5} - \cfrac{2u^3}{3} + 2u + C \\
                    & \hspace*{-6mm} = \cfrac{2x^2\sqrt{x}}{5} - \cfrac{2x\sqrt{x}}{3} + 2\sqrt{x} + C \\
                    & \hspace*{-6mm} = \cfrac{2}{15}\sqrt{x}(3x^2-5x+15) + C
                \end{flalign*}


            \item[(2)]
                \begin{flalign*}
                    & \hspace*{-10mm} \int \cfrac{1}{x}log{x} \ dx \cdots (*) \\
                    & \hspace*{-10mm} u = log{x}とすると \\
                    & \hspace*{-10mm} (*) = \int u du \\
                    & \hspace*{-6mm} = \cfrac{u^2}{2} + C_1 \\
                    & \hspace*{-6mm} = \cfrac{log^2{x}}{2} + C_1
                \end{flalign*}

        \end{multicols}

        \begin{multicols}{2}

            \item[(3)]
                \begin{flalign*}
                    & \hspace*{-10mm} \int x^2\sin{x} \ dx \\
                    & \hspace*{-6mm} = -x^2\cos{x} + 2\int x\cos{x} \ dx + C_2 \\
                    & \hspace*{-6mm} = -x^2\cos{x} + 2x\sin{x} - 2\int \sin{x} \ dx + C_2 \\
                    & \hspace*{-6mm} = -x^2\cos{x} + 2x\sin{x} + 2\cos{x} + C_2 \\
                    & \hspace*{-6mm} = 2x\sin{x} + (2-x^2)\cos{x} + C_2
                \end{flalign*}

            \item[(4)]
                \begin{flalign*}
                    & \hspace*{-10mm} \int\cfrac{e^x}{\sqrt{4-e^{2x}}} \ dx \cdots (*) \\
                    & \hspace*{-10mm} u = e^x とすると \\
                    & \hspace*{-10mm} (*) = \int \cfrac{1}{\sqrt{4-u^2}} \ du \\
                    & \hspace*{-6mm} = \int \cfrac{1}{2\sqrt{1-\cfrac{u^2}{4}}} \ du \\
                    & \hspace*{-6mm} = \cfrac{1}{2}\int \cfrac{1}{\sqrt{1-\cfrac{u^2}{4}}} \ du \cdots (**) \\
                    & \hspace*{-10mm} s = \cfrac{u}{2} とすると \\
                    & \hspace*{-10mm} (**) = \int \cfrac{1}{\sqrt{1-s^2}} \ ds \\
                    & \hspace*{-4mm} = \sin^{-1}{s} + C_3 \\
                    & \hspace*{-4mm} = \sin^{-1}{\cfrac{u}{2}} + C_3 \\
                    & \hspace*{-4mm} = \sin^{-1}{\cfrac{e^x}{2}} + C_3
                \end{flalign*}

        \end{multicols}

        \begin{multicols}{2}

            \item[(5)]
                \begin{flalign*}
                    & \hspace*{-10mm} \int \sin{-1}{x} \ dx \\
                    & \hspace*{-6mm} = x\sin^{-1}{x}  - \int \cfrac{x}{\sqrt{1-x^2}} \ dx \cdots (*) \\
                    & \hspace*{-10mm} u = 1-x^2 とすると \\
                    & \hspace*{-10mm} (*) = x\sin^{-1}{x} + \cfrac{1}{2} \int \cfrac{1}{\sqrt{u}} \ du \\
                    & \hspace*{-6mm} = x\sin^{-1}{x} + \sqrt{u} + C_4 \\
                    & \hspace*{-6mm} = x\sin^{-1}{x} + \sqrt{1-x^2} + C_4
                \end{flalign*}

            \item[(6)]
                \begin{flalign*}
                    & \hspace*{-10mm} \int \cfrac{x^2+3x}{x^2+3x+2} \ dx \\
                    & \hspace*{-6mm} = \int \left(\cfrac{2}{x+2} - \cfrac{2}{x+1} + 1 \right) \ dx \\
                    & \hspace*{-6mm} = 2\int\cfrac{1}{x+2} \ dx - 2\int\cfrac{1}{x+1} \ dx + \int 1 \ dx \\
                    & \hspace*{-6mm} = 2log{x+2} - 2log{x+1} + x + C_5
                \end{flalign*}

        \end{multicols}

        \begin{multicols}{2}

            \item[(7)]
                \begin{flalign*}
                    & \hspace*{-10mm} \int \cfrac{1}{x(x+1)^2} \ dx \\
                    & \hspace*{-6mm} = \int \left(-\cfrac{1}{x+1} - \cfrac{1}{(x+1)^2} + \cfrac{1}{x} \right) \ dx \\
                    & \hspace*{-6mm} = -\int \cfrac{1}{x+1} \ dx - \int \cfrac{1}{(x+1)^2} \ dx + \int \cfrac{1}{x} \ dx \\
                    & \hspace*{-6mm} = log{x} - log{(x+1)} + \cfrac{1}{x+1} + C_6
                \end{flalign*}

            \item[(8)]
                \begin{flalign*}
                    & \hspace*{-10mm} \int \cfrac{12}{x^3+8} \ dx \\
                    & \hspace*{-6mm} = 12 \int \cfrac{1}{x^3+8} \ dx \\
                    & \hspace*{-6mm} = 12 \int \cfrac{1}{(x+2)(x^2-2x+4)} \ dx \\
                    & \hspace*{-6mm} = 12 \int \left(\cfrac{4-x}{12(x^2-2x+4)} + \cfrac{1}{12(x+2)} \right) \ dx \\
                    & \hspace*{-6mm} = \int\cfrac{4-x}{x^2-2x+4} \ dx + \int\cfrac{1}{x+2} \ dx \\
                    & \hspace*{-6mm} = -\cfrac{1}{2}log{(x^2-2x+4)} + log{(x+2)} + \sqrt{3} \tan^{-1}{\cfrac{x-1}{\sqrt{3}}} + C_7
                \end{flalign*}

        \end{multicols}

    \end{description}

    \newpage

    \section*{4}
    次の定積分を求めて下さい。

    \begin{description}
        \setlength{\itemsep}{0.5cm}

        \begin{multicols}{2}

            \item[(1)]
                \begin{flalign*}
                    & \hspace*{-10mm} \int_{0}^{1-e} \cfrac{1}{x-1} \ dx \\
                    & \hspace*{-6mm} = -\int_{1-e}^{0} \cfrac{1}{x-1} \ dx \cdots (*) \\
                    & \hspace*{-10mm} u = x-1  とすると \\
                    & \hspace*{-10mm} (*) = -\int_{-e}^{-1} \cfrac{1}{u} \ du \\
                    & \hspace*{-6mm} = \left[-log{u}\right]_{-e}^{-1} \\
                    & \hspace*{-6mm} = -log{(-1)} + log{(-e)} \\
                    & \hspace*{-6mm} = 1
                \end{flalign*}

            \item[(2)]
                \begin{flalign*}
                    & \hspace*{-10mm} \int_{0}^{1} x^2e^{x^3} \ dx \cdots (*) \\
                    & \hspace*{-10mm} u = x^3 とすると \\
                    & \hspace*{-10mm} (*) = \cfrac{1}{3} \int_{0}^{1} e^u \ du \\
                    & \hspace*{-6mm} = \left[\cfrac{e^u}{3}\right]_{0}^{1} \\
                    & \hspace*{-6mm} = \cfrac{e-1}{3}
                \end{flalign*}

        \end{multicols}

        \begin{multicols}{2}

            \item[(3)]
                \begin{flalign*}
                    & \hspace*{-10mm} \int_{1}^{4} \cfrac{1}{\sqrt{x}} \ log{x} \ dx \\
                    & \hspace*{-6mm} = \left[2\sqrt{x} \ log{x}\right]_{1}^{4} - 2 \int_{1}^{4}\cfrac{1}{\sqrt{x}} \\
                    & \hspace*{-6mm} = 4log{4} - 2\left[2\sqrt{x}\right]_{1}^{4} \\
                    & \hspace*{-6mm} = 4log{4} - 4
                \end{flalign*}

            \item[(4)]
                \begin{flalign*}
                    & \hspace*{-10mm} \int_{-1}^{1} \cfrac{1}{x^2+2x+5} \ dx \\
                    & \hspace*{-6mm} = \int_{-1}^{1} \cfrac{1}{(x+1)^2 + 4} \ dx \cdots (*) \\
                    & \hspace*{-10mm} u = x+1 とすると \\
                    & \hspace*{-10mm} (*) = \int_{0}^{2} \cfrac{1}{u^2+4} \ du \\
                    & \hspace*{-6mm} = \int_{0}^{2} \cfrac{1}{4\left(\cfrac{u^2}{4} + 1 \right)} \ du \\
                    & \hspace*{-6mm} = \cfrac{1}{4} \int_{0}^{2} \cfrac{1}{\cfrac{u^2}{4} + 1} \ du \\
                    & \hspace*{-10mm} s = \cfrac{u}{2} とすると \\
                    & \hspace*{-6mm} = \cfrac{1}{2} \int_{0}^{1} \cfrac{1}{s^2+1} \ ds \\
                    & \hspace*{-6mm} = \cfrac{1}{2}\left[\tan^{-1}{s}\right]_{0}^{1} \\
                    & \hspace*{-6mm} = \cfrac{\pi}{8}
                \end{flalign*}

        \end{multicols}

    \end{description}

    \newpage

    \section*{5}
    関数 $f(x) = x^4-4x^3 $ に対して、次の問いに答えて下さい。

    \begin{description}
        \setlength{\itemsep}{0.5cm}

        \item[(1)] 2次導関数まで求め、極限と変曲点を求めて下さい。

        \begin{multicols}{2}

            \begin{flalign*}
                & \hspace*{-10mm} f^{\prime}(x) = 4x^3 - 12x^2 = 4x^2(x - 3) \\
                & \hspace*{-10mm} f^{\prime\prime}(x) = 12x(x-2) \\
                & \hspace*{-10mm} f^{\prime}(x) \geq 0 \ \Leftrightarrow \ 4x^2(x - 3) \geq 0 \\
                & \hspace*{5mm} \Leftrightarrow \ x = 0, 3 \leq x \\
                & \hspace*{-10mm} f^{\prime\prime}(x) \geq 0 \ \Leftrightarrow \ 12x(x - 2) \geq 0 \\
                & \hspace*{5mm} \Leftrightarrow \ 0 \leq x \leq 2
            \end{flalign*}

            $ \begin{array}{c||c|c|c|c|c|c|c}
                \hline
                x & \cdots & 0 & \cdots & 2 & \cdots & 3 & \cdots \\
                \hline
                f^{\prime}(x) & - & 0 & - & - & - & 0 & + \\
                \hline
                f^{\prime\prime}(x) & + & 0 & - & 0 & + & + & + \\
                \hline
                f(x) & $\cvdec$ & 0 & $\ccdec$ & -16 & $\cvdec$ & -27 & $\cvinc$
            \end{array} $

            $ \left\{
                \begin{array}{l}
                    極大値 : なし \\
                    極小値 : f(3) = -27 \\
                    変曲点 : (0, \ 0), \ (2, \ -16)
                \end{array}
            \right. $

        \end{multicols}

        \item[(2)] 直線 $x-2$ と $x=2$ 間でグラフ $y=f(x)$ と $x軸$ に挟まれた領域の面積を求めて下さい。

            \begin{flalign*}
                & \hspace*{-4mm} \int x^4-4x^3 \ dx = \cfrac{x^5}{5} - x^4 + C \\
                & \hspace*{-4mm} \cfrac{x^5}{5} - x^4 = 0 \ \Leftrightarrow \ x = 0 \\
                & \hspace*{-4mm} S = \int_{-2}^2\left(\cfrac{x^5}{5} - x^4\right)dx \\
                & \hspace*{-2mm} = \cfrac{1}{5}\int_{-2}{2} x^5 \ dx - \int_{-2}{2} x^4 \ dx \\
                & \hspace*{-2mm} = \left[\cfrac{x^5}{5}\right]_{-2}^{2} \\
                & \hspace*{-2mm} = -\cfrac{64}{5} \\
            \end{flalign*}

    \end{description}

    \section*{6}
    次の関数の3次までのマクローリン多項式を求めて下さい。
    \begin{flalign*}
        & \hspace*{-4mm} f(x) = \tan^{-1}{x}
    \end{flalign*}

    \begin{multicols}{2}

        \begin{flalign*}
            & \hspace*{-4mm} f(x) = \tan^{-1}{x} \\
            & \hspace*{-4mm} f(x) = \tan^{-1}{x} \ \rightarrow \ f(0) = 0 \\
            & \hspace*{-4mm} f^{\prime}(x) = \cfrac{1}{x^2+1} \ \rightarrow \ f^{\prime}(0) = 1 \\
            & \hspace*{-4mm} f^{\prime\prime}(x) = -\cfrac{2x}{(x^2+1)^2} \ \rightarrow \ f^{\prime\prime}(0) = 0 \\
            & \hspace*{-4mm} f^{\prime\prime\prime}(x) = \cfrac{6x^2-2}{(x^2+1)^3} \ \rightarrow \ f^{\prime\prime\prime}(0) = -2 \\
            & \hspace*{10mm} \therefore \ P_n(x) = 0 + x + 0 - \cfrac{x^3}{3} = x - \cfrac{x^3}{3}
        \end{flalign*}

    \end{multicols}

\end{document}