\documentclass{jsarticle}

\usepackage{amsmath}
\usepackage{setspace}
\usepackage{nccmath}

\title{基礎数理演習課題3}
\author{21716070 縫嶋慧深}

\begin{document}

	\maketitle

	\section*{0.1}
    次の行列が逆行列を持つか判定し、持つならば逆行列も求めて下さい。

	\begin{description}
		\setlength{\itemsep}{0.5cm}
        \begin{spacing}{1.2}

            \item[(1)]
                $ A $ =
                $ \begin{pmatrix}
                    5 & 9 \\
                    4 & 7
                \end{pmatrix} $

                $ det(A) = 35 - 36 = -1 \neq 0 $

                $ A^{-1} = -\frac{1}{1} $
                $ \begin{pmatrix}
                    7 & -9 \\
                    -4 & 5
                \end{pmatrix} $
                =
                $ \begin{pmatrix}
                    -7 & 9 \\
                    4 & -5
                \end{pmatrix} $

            \item[(2)]
                $ A $ =
                $ \begin{pmatrix}
                    -4 & 2 \\
                    6 & -3
                \end{pmatrix} $

                $ det(A) = 12 - 12 = 0 $

                $ det(A) = 0 $ より、逆行列を持たない。

            \item[(3)]
                $ A $ =
                $ \begin{pmatrix}
                    1 & \sqrt{2} \\
                    \sqrt{2} & 3
                \end{pmatrix} $

                $ det(A) = 3 - 2 = 1 \neq 0 $

                $ A^{-1} = \frac{1}{1} $
                $ \begin{pmatrix}
                    3 & -\sqrt{2} \\
                    -\sqrt{2} & 1
                \end{pmatrix} $
                =
                $ \begin{pmatrix}
                    3 & -\sqrt{2} \\
                    -\sqrt{2} & 1
                \end{pmatrix} $

            \item[(4)]
                $ A $ =
                $ \begin{pmatrix}
                    -\sqrt{3}
                \end{pmatrix} $

                $ det(A) = -\sqrt{3} $

                $ A $との積が単位行列
                $ \begin{pmatrix}
                    1
                \end{pmatrix} $
                になるのは、

                $ A^{-1} = $
                $ \begin{pmatrix}
                    -\frac{1}{\sqrt{3}}
                \end{pmatrix} $

            \item[(5)]
                $ A $ =
                $ \begin{pmatrix}
                    3 & -1 & 4 \\
                    4 & 1 & 3 \\
                    1 & 3 & -2
                \end{pmatrix} $

                $ det(A) = -6 + (-3) + 48 - 4 - 8 - 27 = 0 $

                $ det(A) = 0 $ より、逆行列を持たない。

		\end{spacing}
	\end{description}

	\section*{0.2}
	次の行列が逆行列を持つような実数$ a $を求めてください。

    $ A $ =
    $ \begin{pmatrix}
        a & 1 & 1 \\
        1 & a & 1 \\
        1 & 1 & a
    \end{pmatrix} $

    \begin{fleqn}
        \begin{eqnarray*}
            det(A) & = & a^3 + 1 + 1 - a - a - a \\
            & = & a^3 - 3a + 2
        \end{eqnarray*}
    \end{fleqn}

    $ det(A) \neq 0 $のとき$ A $は逆行列を持つので、

    \begin{fleqn}
        \begin{eqnarray*}
            det(A) & = & a^3 - 3a + 2 \\
            & = & 2
        \end{eqnarray*}
    \end{fleqn}
    と置くと、

    \begin{fleqn}
        \begin{eqnarray*}
            a^3 - 3a + 2 & = & 2 \\
            a^3 - 3a & = & 0 \\
            a(a^2 - 3) & = & 0 \\
            a & = & 0, \pm \sqrt{3}
        \end{eqnarray*}
    \end{fleqn}
\end{document}