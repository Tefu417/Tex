\documentclass[fleqn]{jsarticle}

\usepackage{amsmath, amssymb}
\usepackage{setspace}
\usepackage{nccmath}
\usepackage{type1cm}

\title{基礎数理演習課題4}
\author{21716070 縫嶋慧深}

\begin{document}

	\maketitle

    \section*{0}
    次の行列の関係式について以下をそれぞれ求めて下さい。\\\\
    $ \begin{pmatrix}
        1 & 0 & 2 \\
        0 & 1 & -1 \\
        2 & 1 & 2
    \end{pmatrix} $
    $ \begin{pmatrix}
        x_1 & x_2 & x_3 \\
        y_1 & y_2 & y_3 \\
        z_1 & z_2 & z_3
    \end{pmatrix} $
    =
    $ \begin{pmatrix}
        1 & 0 & 0 \\
        0 & 1 & 0 \\
        0 & 0 & 1
    \end{pmatrix} $\\\\

    \begin{description}
		\setlength{\itemsep}{0.5cm}
        \begin{spacing}{1.2}

            \item[(1)]
                $ x_1, y_1, z_1 $ \\\\
                $ \left(
                    \begin{array}{ccc|c}
                        1 & 0 & 2 & 1 \\
                        0 & 1 & -1 & 0 \\
                        2 & 1 & 2 & 0
                    \end{array}
                \right) $
                $ \xrightarrow{R_3-2R_1} $
                $ \left(
                    \begin{array}{ccc|c}
                        1 & 0 & 0 & 1 \\
                        0 & 1 & 0 & 0 \\
                        0 & 1 & -2 & -2
                    \end{array}
                \right) $
                $ \xrightarrow{R_3-R_2} $
                $ \left(
                    \begin{array}{ccc|c}
                        1 & 0 & 0 & 1 \\
                        0 & 1 & 0 & 0 \\
                        0 & 0 & -2 & -2
                    \end{array}
                \right) $\\
                $ \xrightarrow{R_3\times(-\frac{1}{2})} $
                $ \left(
                    \begin{array}{ccc|c}
                        1 & 0 & 0 & 1 \\
                        0 & 1 & 0 & 0 \\
                        0 & 0 & 1 & 1
                    \end{array}
                \right)
                \hspace{60pt}
                (x_1, y_1, z_1) = (1, 0, 1) $

            \item[(2)]
                $ x_2, y_2, z_2 $ \\\\
                $ \left(
                    \begin{array}{ccc|c}
                        1 & 0 & 2 & 0 \\
                        0 & 1 & -1 & 1 \\
                        2 & 1 & 2 & 0
                    \end{array}
                \right) $
                $ \xrightarrow{R_3-2R_1} $
                $ \left(
                    \begin{array}{ccc|c}
                        1 & 0 & 0 & 0 \\
                        0 & 1 & 0 & 1 \\
                        0 & 1 & -2 & 0
                    \end{array}
                \right) $
                $ \xrightarrow{R_3-R_2} $
                $ \left(
                    \begin{array}{ccc|c}
                        1 & 0 & 0 & 0 \\
                        0 & 1 & 0 & 1 \\
                        0 & 0 & -2 & -1
                    \end{array}
                \right) $\\
                $ \xrightarrow{R_3\times(-\frac{1}{2})} $
                $ \left(
                    \begin{array}{ccc|c}
                        1 & 0 & 0 & 0 \\
                        0 & 1 & 0 & 1 \\
                        0 & 0 & 1 & \frac{1}{2}
                    \end{array}
                \right)
                \hspace{60pt}
                (x_2, y_2, z_2) = (0, 1, \frac{1}{2}) $

            \newpage

            \item[(3)]
                $ x_3, y_3, z_3 $ \\\\
                $ \left(
                    \begin{array}{ccc|c}
                        1 & 0 & 2 & 0 \\
                        0 & 1 & -1 & 0 \\
                        2 & 1 & 2 & 1
                    \end{array}
                \right) $
                $ \xrightarrow{R_3-2R_1} $
                $ \left(
                    \begin{array}{ccc|c}
                        1 & 0 & 0 & 0 \\
                        0 & 1 & 0 & 0 \\
                        0 & 1 & -2 & 1
                    \end{array}
                \right) $
                $ \xrightarrow{R_3-R_2} $
                $ \left(
                    \begin{array}{ccc|c}
                        1 & 0 & 0 & 0 \\
                        0 & 1 & 0 & 0 \\
                        0 & 0 & -2 & 1
                    \end{array}
                \right) $\\
                $ \xrightarrow{R_3\times(-\frac{1}{2})} $
                $ \left(
                    \begin{array}{ccc|c}
                        1 & 0 & 0 & 0 \\
                        0 & 1 & 0 & 1 \\
                        0 & 0 & 1 & -\frac{1}{2}
                    \end{array}
                \right)
                \hspace{60pt}
                (x_3, y_3, z_3) = (0, 0, -\frac{1}{2}) $

        \end{spacing}
    \end{description}

    \section{}
    次の連立一次方程式の解を、行列の基本変形を用いて求めて下さい。\\

    \begin{description}
		\setlength{\itemsep}{0.5cm}
        \begin{spacing}{1.2}

            \item[(1)]
                $ \left\{
                    \begin{array}{l}
                        4x + 3y = 0 \\
                        3x + 2y = -1 \\
                        -2x - y = 2
                    \end{array}
                \right. $ \\\\
                拡大係数行列を簡約化する。\\\\
                $ \left(
                    \begin{array}{cc|c}
                        4 & 3 & 0 \\
                        3 & 2 & -1 \\
                        -2 & -1 & 2
                    \end{array}
                \right)
                \xrightarrow{R_1 \times \frac{1}{4}}
                \left(
                    \begin{array}{cc|c}
                        1 & \frac{3}{4} & 0 \\
                        3 & 2 & -1 \\
                        -2 & -1 & 2
                    \end{array}
                \right)
                \xrightarrow{\substack{R_2-R_1 \times 3 \\ R_3+R_1 \times 2}}
                \left(
                    \begin{array}{cc|c}
                        1 & \frac{3}{4} & 0 \\
                        0 & -\frac{1}{4} & -1 \\
                        0 & \frac{1}{2} & 2
                    \end{array}
                \right)\\
                \xrightarrow{\substack{R_2 \times (-4) \\ R_3 \times 2}}
                \left(
                    \begin{array}{cc|c}
                        1 & \frac{3}{4} & 0 \\
                        0 & 1 & 4 \\
                        0 & 1 & 4
                    \end{array}
                \right)
                \xrightarrow{\substack{R_1-R_2 \times \frac{3}{4} \\ R_3-R_2}}
                \left(
                    \begin{array}{cc|c}
                        1 & 0 & -3 \\
                        0 & 1 & 4 \\
                        0 & 0 & 0
                    \end{array}
                \right) $ \\\\
                連立方程式に戻すと、
                $ \left\{
                    \begin{array}{l}
                        x = -3 \\
                        y = 4
                    \end{array}
                \right. $

            \item[(2)]
                $ \left\{
                    \begin{array}{l}
                        -2x + y + z = 2 \\
                        x - 2y + z = 2 \\
                        x + y - 2z = -4
                    \end{array}
                \right. $ \\\\
                拡大係数行列を簡約化する。\\\\
                $ \left(
                    \begin{array}{ccc|c}
                        -2 & 1 & 1 & 2 \\
                        1 & -2 & 1 & 2 \\
                        1 & 1 & -2 & -4
                    \end{array}
                \right)
                \xrightarrow{R_1 \times -\frac{1}{2}}
                \left(
                    \begin{array}{ccc|c}
                        1 & -\frac{1}{2} & -\frac{1}{2} & -1 \\
                        1 & -2 & 1 & 2 \\
                        1 & 1 & -2 & -4
                    \end{array}
                \right)
                \xrightarrow{\substack{R_2-R_1 \\ R_3-R_1}}
                \left(
                    \begin{array}{ccc|c}
                        1 & -\frac{1}{2} & -\frac{1}{2} & -1 \\
                        0 & -\frac{3}{2} & \frac{3}{2} & 3 \\
                        0 & \frac{3}{2} & -\frac{3}{2} & -3
                    \end{array}
                \right)\\
                \xrightarrow{R_3-R_2}
                \left(
                    \begin{array}{ccc|c}
                        1 & -\frac{1}{2} & -\frac{1}{2} & -1 \\
                        0 & -\frac{3}{2} & \frac{3}{2} & 3 \\
                        0 & 0 & 0 & 0
                    \end{array}
                \right)
                \xrightarrow{R_2 \times (-\frac{2}{3})}
                \left(
                    \begin{array}{ccc|c}
                        1 & -\frac{1}{2} & -\frac{1}{2} & -1 \\
                        0 & 1 & -1 & -2 \\
                        0 & 0 & 0 & 0
                    \end{array}
                \right)
                \xrightarrow{R_1+R_2 \times \frac{1}{2}}
                \left(
                    \begin{array}{ccc|c}
                        1 & 0 & -1 & -2 \\
                        0 & 1 & -1 & -2 \\
                        0 & 0 & 0 & 0
                    \end{array}
                \right) $ \\\\
                連立方程式に戻すと、
                $ \left\{
                    \begin{array}{l}
                        x - z = -2 \\
                        y - z = -2
                    \end{array}
                \right. $ \\\\
                主成分を左辺に残し、残りを右辺に移行する。
                $ \left\{
                    \begin{array}{l}
                        x = z - 2 \\
                        y = z - 2
                    \end{array}
                \right. $ \\\\
                $z$ を $s \in \mathbb{R}$ とおくと、
                $ \left\{
                    \begin{array}{l}
                        x = s - 2 \\
                        y = s - 2 \\
                        z = s
                    \end{array}
                \right.
                (s \in \mathbb{R}) $

            \item[(3)]
                $ \left\{
                    \begin{array}{l}
                        x - y + z + w = 1 \\
                        2x - 2y - z + 2w = -1 \\
                        3x - 3y + z + 3w = 1
                    \end{array}
                \right. $ \\\\
                拡大係数行列を簡約化する。\\\\
                $ \left(
                    \begin{array}{cccc|c}
                        1 & -1 & 1 & 1 & 1 \\
                        2 & -2 & -1 & 2 & -1 \\
                        3 & -3 & 1 & 3 & 1
                    \end{array}
                \right)
                \xrightarrow{\substack{R_2-2R_1 \\ R_3-3R_1}}
                \left(
                    \begin{array}{cccc|c}
                        1 & -1 & 1 & 1 & 1 \\
                        0 & 0 & -3 & 0 & -3 \\
                        0 & 0 & -2 & 0 & -2
                    \end{array}
                \right)
                \xrightarrow{\substack{R_2 \times (-\frac{1}{3}) \\ R_3 \times (-\frac{1}{2})}}
                \left(
                    \begin{array}{cccc|c}
                        1 & -1 & 1 & 1 & 1 \\
                        0 & 0 & 1 & 0 & 1 \\
                        0 & 0 & 1 & 0 & 1
                    \end{array}
                \right)\\
                \xrightarrow{\substack{R_1-R_2 \\ R_3-R_2}}
                \left(
                    \begin{array}{cccc|c}
                        1 & -1 & 0 & 1 & 0 \\
                        0 & 0 & 1 & 0 & 1 \\
                        0 & 0 & 0 & 0 & 0
                    \end{array}
                \right)$ \\\\
                連立方程式に戻すと、
                $ \left\{
                    \begin{array}{l}
                        x - y + w = 0 \\
                        z = 1
                    \end{array}
                \right. $ \\\\
                主成分 $x, z$ を左辺に残し、残りを右辺に移行する。
                $ \left\{
                    \begin{array}{l}
                        x = y - w \\
                        z = 1
                    \end{array}
                \right. $ \\\\
                $y, w$ をそれぞれ $s, t \in \mathbb{R}$ とおくと、
                $ \left\{
                    \begin{array}{l}
                        x = s - t \\
                        y = s \\
                        z = 1 \\
                        w = t
                    \end{array}
                \right.
                (s, t \in \mathbb{R}) $

        \end{spacing}
    \end{description}

\end{document}