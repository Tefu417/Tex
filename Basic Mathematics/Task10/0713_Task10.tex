\documentclass[fleqn]{jsarticle}
\setlength{\headsep}{15truemm}

\usepackage{amsmath, amssymb}
\usepackage{setspace}
\usepackage{nccmath}
\usepackage{type1cm}
\usepackage{multicol}

\title{\vspace{-50mm} 基礎数理演習課題10}
\author{21716070 縫嶋慧深}

\begin{document}

	\maketitle

    \section*{0}
    次の関数を2次導関数まで求め、極値を求めて下さい。また、変曲点を求めて下さい。

    \begin{description}
        \setlength{\itemsep}{0.5cm}

        \begin{multicols}{2}

            \item[(1)]
                \begin{flalign*}
                    & \hspace*{-10mm} f(x) = \cfrac{x}{e^x} \\
                    & \hspace*{-10mm} f^{\prime}(x) = \cfrac{e^x - x \cdot e^x}{e^{2x}} = \cfrac{1 - x}{e^x} \\
                    & \hspace*{-10mm} f^{\prime\prime}(x) = \cfrac{-e^x - (1 - x)e^x}{e^{2x}} = \cfrac{-2 + x}{e^x} \\
                    & \hspace*{-10mm} f^{\prime}(x) \geq 0 \ \Leftrightarrow \ \cfrac{1 - x}{e^x} \geq 0 \\
                    & \hspace*{5mm} \Leftrightarrow \ x \leq 1 \\
                    & \hspace*{-10mm} f^{\prime\prime}(x) \geq 0 \ \Leftrightarrow \ \cfrac{-2 + x}{e^x} \geq 0 \\
                    & \hspace*{5mm} \Leftrightarrow \ x \geq 2
                \end{flalign*}

                $ \left\{
                    \begin{array}{l}
                        極大値 : f(1) = \frac{1}{e} \\
                        極小値 : なし
                        変曲点 : \left(2, \ \cfrac{x}{e^x}\right)
                    \end{array}
                \right. $

            \item[(2)]
                \begin{flalign*}
                    & \hspace*{-10mm} f(x) = x^5 - 5x^4 \\
                    & \hspace*{-10mm} f^{\prime}(x) = 5x^4 - 20x^3 = 5x^3(x - 4) \\
                    & \hspace*{-10mm} f^{\prime\prime}(x) = 20x^3 - 60x^2 = 20x^2(x - 3) \\
                    & \hspace*{-10mm} f^{\prime}(x) \geq 0 \ \Leftrightarrow \ 5x^3(x - 4) \geq 0 \\
                    & \hspace*{5mm} \Leftrightarrow \ x \leq 0, 4 \leq x \\
                    & \hspace*{-10mm} f^{\prime\prime}(x) \geq 0 \ \Leftrightarrow \ 20x^2(x - 3) \geq 0 \\
                    & \hspace*{5mm} \Leftrightarrow \ x \geq 3
                \end{flalign*}

                $ \left\{
                    \begin{array}{l}
                        極大値 : f(0) = 0 \\
                        極小値 : f(4) = -256 \\
                        変曲点 : (3, -162)
                    \end{array}
                \right. $

        \end{multicols}

    \end{description}


    \vspace{-5mm} \section*{1}
    次の不定積分を求めて下さい。(積分定数として$C, C_1, C_2, \cdots $を断らずに用いてよい)

    \begin{description}
        \setlength{\itemsep}{0.5cm}

        \begin{multicols}{2}

            \item[(1)]
                \begin{flalign*}
                    & \hspace*{-10mm} \int(-\sin{x} - \cos{x})dx \\
                    & \hspace*{2mm} = \cos{x} - \sin{x} + C
                \end{flalign*}


            \item[(2)]
                \begin{flalign*}
                    & \hspace*{-10mm} \int(4 + x^4 + 4^x)dx \\
                    & \hspace*{2mm} = 4x + \cfrac{x^5}{5} + \cfrac{4^x}{log{4}} + C_1
                \end{flalign*}

        \end{multicols}

        \begin{multicols}{2}

            \item[(3)]
                \begin{flalign*}
                    & \hspace*{-10mm} \int(x^{\frac{2}{3}} + x^{-\frac{2}{3}} + x^{-\frac{3}{2}})dx \\
                    & \hspace*{2mm} = \cfrac{3x^{\frac{5}{3}}}{5} + 3x^{\frac{1}{3}} - 2x^{\left(-\frac{1}{2}\right)} + C_2 \\
                    & \hspace*{2mm} = \cfrac{3x^{\frac{5}{3}}}{5} + 3\sqrt[3]{x} - \cfrac{2}{\sqrt{x}} + C_2
                \end{flalign*}

            \item[(4)]
                \begin{flalign*}
                    & \hspace*{-10mm} \int(\sqrt{x} + \cfrac{1}{\sqrt{x}})dx \\
                    & \hspace*{2mm} = \cfrac{2x^{\frac{3}{2}}}{3} + 2\sqrt{x} + C_3 \\
                    & \hspace*{2mm} = \cfrac{2}{3}\sqrt{x}(x + 3) + C_3
                \end{flalign*}

        \end{multicols}

        \item[(5)]
            \begin{flalign*}
                & \hspace*{-10mm} \int\left(\frac{1}{x} + \frac{1}{x^2} + \frac{1}{x^2 + 1}\right)dx \\
                & \hspace*{2mm} = log{x} -\frac{1}{x} + \tan^{-1}{x} + C_4
            \end{flalign*}

    \end{description}

    \section*{2}
    次の定積分を求めて下さい。

    \begin{description}
        \setlength{\itemsep}{0.5cm}

        \begin{multicols}{2}

            \item[(1)]
                \begin{flalign*}
                    & \hspace*{-10mm} \int_0^2(6x^2 - 6x + 6)dx \\
                    & \hspace*{-2mm} = \left[2x^3 - 3x^2 + 6x\right]_0^2 \\
                    & \hspace*{-2mm} = (16 - 12 + 12) - 0 = 16
                \end{flalign*}

            \item[(2)]
                \begin{flalign*}
                    & \hspace*{-10mm} \int_1^2(2^x - x^2)dx \\
                    & \hspace*{-2mm} = \left[\cfrac{2^x}{log{2}} - \cfrac{x^3}{3}\right]_1^2 \\
                    & \hspace*{-2mm} = \left(\cfrac{4}{log{2}} - \frac{8}{3}\right) - \left(\cfrac{2}{log{2}} - \frac{1}{3}\right) \\
                    & \hspace*{-2mm} = \cfrac{2}{log{2}} - \frac{7}{3}
                \end{flalign*}

        \end{multicols}

        \begin{multicols}{2}

            \item[(3)]
                \begin{flalign*}
                    & \hspace*{-10mm} \int_0^{\frac{\pi}{4}}\cfrac{1}{\cos^2{x}} \ dx \\
                    & \hspace*{-2mm} = \left[\tan{x}\right]_0^{\frac{\pi}{4}} \\
                    & \hspace*{-2mm} = 1 - 0 = 0
                \end{flalign*}

            \item[(4)]
                \begin{flalign*}
                    & \hspace*{-10mm} \int_{-\frac{1}{2}}^{\frac{1}{2}}\cfrac{1}{\sqrt{1 - x^2}} \ dx \\
                    & \hspace*{-2mm} = \left[\sin^{-1}{x}\right]_{-\frac{1}{2}}^{\frac{1}{2}} \\
                    & \hspace*{-2mm} = \cfrac{\pi}{6} - \left(-\cfrac{\pi}{6}\right) = \cfrac{\pi}{3}
                \end{flalign*}

        \end{multicols}

    \end{description}

    \section*{3}
    $x = -1$ と $x = 1$ の間でグラフ $y = e^x - 1$ と $x軸$ に挟まれた領域の(通常の)面積を求めて下さい。

    \begin{flalign*}
        & \hspace*{-4mm} \int(e^x-1)dx = e^x - x + C \\
        & \hspace*{-4mm} e^x-1 = 0 \ \Leftrightarrow \ x = 0 \\
        & \hspace*{-4mm} S = \int_{-1}^1(e^x-1)dx = -\int_{-1}^0(e^x-1)dx + \int_0^1(e^x-1)dx \\
        & \hspace*{25mm} = -\left[e^x-x\right]_{-1}^0 + \left[e^x-x\right]_0^1 \\
        & \hspace*{25mm} = -(1 - (e^{-1} + 1)) + ((e - 1) - 1) \\
        & \hspace*{25mm} = e + \cfrac{1}{e} - 2 \\
    \end{flalign*}

\end{document}