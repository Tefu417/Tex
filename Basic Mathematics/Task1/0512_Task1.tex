% テンプレート(スタイル)を指定する

\documentclass{jsarticle}

% パッケージ
\usepackage{amsmath}

\title{基礎数理演習課題1}
\author{縫嶋慧深}

\begin{document}

	\maketitle

	\section*{0.1}
	二つの空間ベクトル$\overrightarrow{a}= (1, 1, -4)$と$\overrightarrow{b}= (0, 3, 3)$のなす角を求めて下さい。

	\[
		cos\theta
			= \frac {\overrightarrow{a} \cdot \overrightarrow{b}}
				{|\overrightarrow{a}| |\overrightarrow{b}|}\
			= \frac {a_1b_1+a_2b_2+a_3b_3}
				{\sqrt {{a_1}^2+{a_2}^2+{a_3}^2} \sqrt {{b_1}^2+{b_2}^2+{b_3}^2}}
	\]

	\[
		cos\theta
			= \frac {0 + 0 + (-12)} {\sqrt {1 + 1 + 16} \sqrt {0 + 9 + 9}}\
			= - \frac {12}{18}\
			= - \frac{2}{3}
	\]

	\[
		\theta = 120^\circ
	\]

	\section*{0.2}
	平面ベクトル$\overrightarrow{a}= (2, -1)$と$\overrightarrow{b}= (1, -5)$と$\overrightarrow{c}= (-1, 2)$に大して、関係式$\overrightarrow{a} = s\overrightarrow{b} + t\overrightarrow{c}$を満たすような実数の組$s, t$を求めて下さい。

	$(2, -1) = s(1, -5) + t(-1, 2)$より、

	\[
		\begin{cases}
			2 = s - t \\
			-1 = -5s + 2t
		\end{cases}
	\]

	\begin{eqnarray*}
		s &=& t + 2 \\
		5s &=& 2t + 1 \\
		\\
		5(t + 2) &=& 2t + 1 \\
		3t &=& -9 \\
		t &=& -3 \\
		s &=& -1
	\end{eqnarray*}

	\[
		(s, t) = (-1, -3)
	\]

	\section{}
	次の行列の計算をして下さい。

	\begin{description}
		\setlength{\itemsep}{0.5cm}

		\item[(1)]
			$ \begin{pmatrix}
				2 & 4 \\
				-2 & -3
			\end{pmatrix} $
			-
			$ \begin{pmatrix}
				1 & -2 \\
				2 & -3
			\end{pmatrix} $
			=
			$ \begin{pmatrix}
				1 & 6 \\
				-4 & 0
			\end{pmatrix} $

		\item[(2)]
			$ 2 \begin{pmatrix}
				3 & -2 & -1 \\
				5 & 6 &-4
			\end{pmatrix} $
			+
			$ \begin{pmatrix}
				1 & -5 & 3 \\
				-2 & -1 & -1
			\end{pmatrix} $
			=
			$ \begin{pmatrix}
				7 & -9 & 1 \\
				8 & 11 & -9
			\end{pmatrix} $

		\item[(3)]
			$ \begin{pmatrix}
				1 & -5 \\
				3 & 2 \\
				-1 & -1
			\end{pmatrix} $
			$ \begin{pmatrix}
				3 & -2 & -1 \\
				-1 & 1 & -2
			\end{pmatrix} $
			=
			$ \begin{pmatrix}
				8 & -7 & 9 \\
				7 & -4 & -7 \\
				-2 & 1 & 3
			\end{pmatrix} $

		\item[(4)]
			$ \begin{pmatrix}
				3 & -2 & -1 \\
				-1 & 1 & -2
			\end{pmatrix} $
			$ \begin{pmatrix}
				1 & -5 \\
				3 & 2 \\
				-1 & -1
			\end{pmatrix} $
			=
			$ \begin{pmatrix}
				-2 & -18 \\
				4 & 9
			\end{pmatrix} $

		\item[(5)]
			$ \begin{pmatrix}
				3 & 5 \\
				1 & 2
			\end{pmatrix} $
			$ \begin{pmatrix}
				2 & -5 \\
				-1 & 3
			\end{pmatrix} $
			=
			$ \begin{pmatrix}
				1 & 0 \\
				-0 & 1
			\end{pmatrix} $

		\item[(6)]
			$ \begin{pmatrix}
				-1 & 2 \\
				2 & 1 \\
				1 & -1
			\end{pmatrix} $
			$ \begin{pmatrix}
				4 & -1 \\
				-3 & 1
			\end{pmatrix} $
			=
			$ \begin{pmatrix}
				-10 & 3 \\
				5 & -1 \\
				7 & -2
			\end{pmatrix} $
	\end{description}



\end{document}