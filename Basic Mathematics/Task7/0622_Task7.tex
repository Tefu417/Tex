\documentclass[fleqn]{jsarticle}

\usepackage{amsmath, amssymb}
\usepackage{setspace}
\usepackage{nccmath}
\usepackage{type1cm}
\usepackage{multicol}

\title{基礎数理演習課題7}
\author{21716070 縫嶋慧深}

\begin{document}

	\maketitle

    \section*{0}
    次で満たされる関数$f$を微分して下さい。
    \begin{description}
		\setlength{\itemsep}{0.5cm}

            \item[(1)]
                $ f(x) = x^2 + \cfrac{1}{x} + \sqrt{x} $
                \begin{eqnarray*}
                    f^{\prime}(x) = 2x - \frac{1}{x^2} + \frac{1}{2\sqrt{x}}
                \end{eqnarray*}

            \item[(2)]
                $ f(x) = e^x\cos{x} $
                \begin{eqnarray*}
                    f^{\prime}(x) &=& e^x\cos{x} + e^x\cdot(-\sin{x}) \\
                    &=& e^x\cos{x} - e^x\sin{x}
                \end{eqnarray*}

            \item[(3)]
                $ f(x) = log(x^4 + 1) $
                \begin{eqnarray*}
                    f^{\prime}(x) = \frac{4x^3}{x^4 + 1}
                \end{eqnarray*}

    \end{description}

    \section*{1}
    次で定義される関数$f$の(自然な)定義域を、区間の記号を用いて表して下さい。

    \begin{multicols}{2}
        \begin{description}
            \setlength{\itemsep}{0.5cm}

                \item[(1)]
                    $ f(x) = log_2(x-9) \\
                    \mathcal{D}(f) = (9, +\infty) $

                \item[(2)]
                    $ f(x) = \sqrt{x - 7} \\
                    \mathcal{D}(f) = [7, +\infty) $

                \item[(3)]
                    $ f(x) = \sqrt{4 - x^2} \\
                    \mathcal{D}(f) = [-2, 2] $

                \item[(4)]
                    $ f(x) = \cfrac{1}{\sqrt{-3-x}} \\
                    \mathcal{D}(f) = (-\infty, -3) $

        \end{description}
    \end{multicols}

    \newpage

    \section*{2}
    次の値を求めて下さい。

    \begin{description}
        \setlength{\itemsep}{0.5cm}

        \begin{multicols}{2}

            \item[(1)]
                $ \sin^{-1}{\cfrac{1}{\sqrt{2}}} $ \\
                $ \sin^{-1}{\cfrac{1}{\sqrt{2}}} = \cfrac{\pi}{4}\left(\in\left[-\frac{\pi}{2}, \frac{\pi}{2}\right]\right) $

            \item[(2)]
                $ \sin^{-1}{\left(-\cfrac{1}{2}\right)} $ \\
                $ \sin^{-1}{\left(-\cfrac{1}{2}\right)} = -\cfrac{\pi}{6}\left(\in\left[-\frac{\pi}{2}, \frac{\pi}{2}\right]\right) $

        \end{multicols}

        \begin{multicols}{2}

            \item[(3)]
                $ \cos^{-1}{\left(-\cfrac{1}{2}\right)} $ \\
                $ \cos^{-1}{\left(-\cfrac{1}{2}\right)} = \cfrac{2\pi}{3}\left(\in\left[0, \pi\right]\right) $

            \item[(4)]
                $ \cos^{-1}{0} $ \\
                $ \cos^{-1}{0} = \cfrac{\pi}{2}\left(\in\left[0, \pi\right]\right) $

        \end{multicols}

        \begin{multicols}{2}

            \item[(5)]
                $ \tan^{-1}{(-1)} $ \\
                $ \tan^{-1}{(-1)} = -\cfrac{\pi}{4}\left(\in\left(-\frac{\pi}{2}, \frac{\pi}{2}\right)\right) $

            \item[(6)]
                $ \tan^{-1}{\cfrac{1}{\sqrt{3}}} $ \\
                $ \tan^{-1}{\cfrac{1}{\sqrt{3}}} = \cfrac{\pi}{6}\left(\in\left(-\frac{\pi}{2}, \frac{\pi}{2}\right)\right) $

        \end{multicols}

    \end{description}


    \section*{3}
    次の方程式を満たす$x$を求めて下さい。

    \begin{description}
		\setlength{\itemsep}{0.5cm}

            \item[(1)]
                $ \cos^{-1}{x} = \sin^{-1}{\cfrac{\sqrt{3}}{2}} $ \\
                $ \sin^{-1}{\cfrac{\sqrt{3}}{2}} = \cfrac{\pi}{3}\left(\in\left[-\frac{\pi}{2}, \frac{\pi}{2}\right]\right) \\
                \cos^{-1}{x} = \cfrac{\pi}{3}\left(\in\left[-\frac{\pi}{2}, \frac{\pi}{2}\right]\right) \\
                x = \cfrac{1}{2} $

            \item[(2)]
                $ \sin^{-1}{x} = \tan^{-1}{1} $
                $ \tan^{-1}{1} = \cfrac{\pi}{4}\left(\in\left(-\frac{\pi}{2}, \frac{\pi}{2}\right)\right) \\
                \sin^{-1}{x} = \cfrac{\pi}{4}\left(\in\left(-\frac{\pi}{2}, \frac{\pi}{2}\right)\right) \\
                x = -\cfrac{1}{\sqrt{2}} $

    \end{description}

\end{document}