\documentclass[fleqn]{jsarticle}

\usepackage{amsmath, amssymb}
\usepackage{setspace}
\usepackage{nccmath}
\usepackage{type1cm}

\title{基礎数理演習課題5}
\author{21716070 縫嶋慧深}

\begin{document}

	\maketitle

    \section*{1}
    次の連立一次方程式の解を求めて下さい。解を持たない場合はその判定もして下さい。\\\\
    \begin{description}
		\setlength{\itemsep}{0.5cm}
        \begin{spacing}{1.2}

            \item[(1)]
                $ \left\{
                    \begin{array}{l}
                        x + 4y = 5 \\
                        -5x + 6y = -2 \\
                        2x + 9y = 9
                    \end{array}
                \right. $ \\\\
                拡大係数行列を簡約化する。\\\\
                $ \left(
                    \begin{array}{cc|c}
                        1 & 4 & 5 \\
                        -5 & 6 & -2 \\
                        2 & 9 & 9
                    \end{array}
                \right)
                \longleftrightarrow
                \left(
                    \begin{array}{cc|c}
                        1 & 4 & 5 \\
                        0 & 26 & 23 \\
                        0 & 1 & -1
                    \end{array}
                \right)
                \longleftrightarrow
                \left(
                    \begin{array}{cc|c}
                        1 & 0 & 9 \\
                        0 & 0 & 49 \\
                        0 & 1 & -1
                    \end{array}
                \right)
                \longleftrightarrow
                \left(
                    \begin{array}{cc|c}
                        1 & 0 & 9 \\
                        0 & 1 & -1 \\
                        0 & 0 & 49
                    \end{array}
                \right) \\
                \longleftrightarrow
                \left(
                    \begin{array}{cc|c}
                        1 & 0 & 9 \\
                        0 & 1 & -1 \\
                        0 & 0 & 1
                    \end{array}
                \right)$\\\\
                拡大係数行列の階数は$3$、係数行列の階数は$2$なので、両者は一致せず、解を持たない。

            \item[(2)]
                $ \left\{
                    \begin{array}{l}
                        2x + y - 3z = 1 \\
                        -3x + 2y + z = 2
                    \end{array}
                \right. $ \\\\
                拡大係数行列を簡約化する。\\\\
                $ \left(
                    \begin{array}{ccc|c}
                        2 & 1 & -3 & 1 \\
                        -3 & 2 & 1 & 2
                    \end{array}
                \right)
                \longleftrightarrow
                \left(
                    \begin{array}{ccc|c}
                        -1 & 3 & -2 & 3 \\
                        2 & 1 & -3 & 1
                    \end{array}
                \right)
                \longleftrightarrow
                \left(
                    \begin{array}{ccc|c}
                        1 & -3 & 2 & -3 \\
                        2 & 1 & -3 & 1
                    \end{array}
                \right)
                \longleftrightarrow
                \left(
                    \begin{array}{ccc|c}
                        1 & -3 & 2 & -3 \\
                        0 & 7 & -7 & 7
                    \end{array}
                \right) \\
                \longleftrightarrow
                \left(
                    \begin{array}{ccc|c}
                        1 & -3 & 2 & -3 \\
                        0 & 1 & -1 & 1
                    \end{array}
                \right)
                \longleftrightarrow
                \left(
                    \begin{array}{ccc|c}
                        1 & 0 & -1 & 0 \\
                        0 & 1 & -1 & 1
                    \end{array}
                \right)$\\\\
                連立方程式に戻すと、
                $ \left\{
                    \begin{array}{l}
                        x - z = 0 \\
                        y - z = 1
                    \end{array}
                \right. $\\\\
                主成分を左辺に残し、残りを右辺に移行する。
                $ \left\{
                    \begin{array}{l}
                        x = z \\
                        y = z + 1
                    \end{array}
                \right. $ \\\\
                $z$ を $s \in \mathbb{R}$ とおくと、
                $ \left\{
                    \begin{array}{l}
                        x = s \\
                        y = s + 1 \\
                        z = s
                    \end{array}
                \right.
                (s \in \mathbb{R}) $

            \item[(3)]
                $ \left\{
                    \begin{array}{l}
                        3x + 2y - 9z = 4 \\
                        x - 2y + 5z = 0 \\
                        2x - y + z = 3
                    \end{array}
                \right. $ \\\\
                拡大係数行列を簡約化する。\\\\
                $ \left(
                    \begin{array}{ccc|c}
                        3 & 2 & -9 & 4 \\
                        1 & -2 & 5 & 0 \\
                        2 & -1 & 1 & 3
                    \end{array}
                \right)
                \longleftrightarrow
                \left(
                    \begin{array}{ccc|c}
                        1 & 3 & -10 & 1 \\
                        1 & -2 & 5 & 0 \\
                        2 & -1 & 1 & 3
                    \end{array}
                \right)
                \longleftrightarrow
                \left(
                    \begin{array}{ccc|c}
                        1 & 3 & -10 & 1 \\
                        0 & -5 & 15 & -1 \\
                        0 & -7 & 21 & 1
                    \end{array}
                \right)
                \longleftrightarrow
                \left(
                    \begin{array}{ccc|c}
                        1 & 3 & -10 & 1 \\
                        0 & 2 & -6 & -2 \\
                        0 & -7 & 21 & 1
                    \end{array}
                \right) \\
                \longleftrightarrow
                \left(
                    \begin{array}{ccc|c}
                        1 & 3 & -10 & 1 \\
                        0 & 1 & -3 & -1 \\
                        0 & -7 & 21 & 1
                    \end{array}
                \right)
                \longleftrightarrow
                \left(
                    \begin{array}{ccc|c}
                        1 & 3 & -1 & 4 \\
                        0 & 1 & -3 & -1 \\
                        0 & 0 & 0 & -6
                    \end{array}
                \right)$ \\\\
                拡大係数行列の階数は$3$、係数行列の階数は$2$なので、両者は一致せず、解を持たない。
        \end{spacing}
    \end{description}

    \section*{2}
    次の行列の階数を求めて下さい。\\\\
    $ A =
    \left(
        \begin{array}{cccc}
            1 & 2 & -1 & 4 \\
            -2 & 3 & 2 & -5 \\
            -1 & 5 & 1 & -1 \\
            4 & 2 & -3 & 12
        \end{array}
    \right) $\\\\
    $ \left(
        \begin{array}{cccc}
            1 & 2 & -1 & 4 \\
            -2 & 3 & 2 & -5 \\
            -1 & 5 & 1 & -1 \\
            4 & 2 & -3 & 12
        \end{array}
    \right)
    \longleftrightarrow
    \left(
        \begin{array}{cccc}
            1 & 2 & -1 & 4 \\
            0 & 7 & 0 & 3 \\
            0 & 7 & 0 & 3 \\
            0 & -6 & 1 & -4
        \end{array}
    \right)
    \longleftrightarrow
    \left(
        \begin{array}{cccc}
            1 & 2 & -1 & 4 \\
            0 & 7 & 0 & 3 \\
            0 & 0 & 0 & 0 \\
            0 & -6 & 1 & -4
        \end{array}
    \right)
    \longleftrightarrow
    \left(
        \begin{array}{cccc}
            1 & 2 & -1 & 4 \\
            0 & 1 & 1 & -1 \\
            0 & 0 & 0 & 0 \\
            0 & -6 & 1 & -4
        \end{array}
    \right) \\
    \longleftrightarrow
    \left(
        \begin{array}{cccc}
            1 & 0 & -3 & 6 \\
            0 & 1 & 1 & -1 \\
            0 & 0 & 0 & 0 \\
            0 & 0 & 7 & -10
        \end{array}
    \right)
    \longleftrightarrow
    \left(
        \begin{array}{cccc}
            1 & 0 & -3 & 6 \\
            0 & 1 & 1 & -1 \\
            0 & 0 & 7 & -10 \\
            0 & 0 & 0 & 0
        \end{array}
    \right)
    \longleftrightarrow
    \left(
        \begin{array}{cccc}
            1 & 0 & -3 & 6 \\
            0 & 1 & 1 & -1 \\
            0 & 0 & 1 & -\frac{10}{7} \\
            0 & 0 & 0 & 0
        \end{array}
    \right)
    \longleftrightarrow
    \left(
        \begin{array}{cccc}
            1 & 0 & 0 & \frac{12}{7} \\
            0 & 1 & 0 & \frac{3}{7} \\
            0 & 0 & 1 & -\frac{10}{7} \\
            0 & 0 & 0 & 0
        \end{array}
    \right) $ \\\\
    行列$A$の階数は$3$ \\\\

    \section*{3}
    次の行列の逆行列を求めて下さい。\\\\

    \begin{description}
		\setlength{\itemsep}{0.5cm}
        \begin{spacing}{1.2}

            \item[(1)]
                $ A =
                \left(
                    \begin{array}{ccc}
                        1 & 2 & 3 \\
                        1 & 1 & 2 \\
                        1 & 2 & 2
                    \end{array}
                \right) $ \\\\
                $ \left(
                    \begin{array}{c|c}
                        A & E_n
                    \end{array}
                \right)
                =
                \left(
                    \begin{array}{ccc|ccc}
                        1 & 2 & 3 & 1 & 0 & 0 \\
                        1 & 1 & 2 & 0 & 1 & 0 \\
                        1 & 2 & 2 & 0 & 0 & 1
                    \end{array}
                \right) $ \\
                $ \longleftrightarrow
                \left(
                    \begin{array}{ccc|ccc}
                        1 & 2 & 3 & 1 & 0 & 0 \\
                        0 & -1 & -1 & -1 & 1 & 0 \\
                        0 & 0 & -1 & -1 & 0 & 1
                    \end{array}
                \right)
                \longleftrightarrow
                \left(
                    \begin{array}{ccc|ccc}
                        1 & 2 & 3 & 1 & 0 & 0 \\
                        0 & 1 & 1 & 1 & -1 & 0 \\
                        0 & 0 & 1 & 1 & 0 & -1
                    \end{array}
                \right)
                \longleftrightarrow
                \left(
                    \begin{array}{ccc|ccc}
                        1 & 0 & 1 & -1 & 2 & 0 \\
                        0 & 1 & 1 & 1 & -1 & 0 \\
                        0 & 0 & 1 & 1 & 0 & -1
                    \end{array}
                \right) \\
                \longleftrightarrow
                \left(
                    \begin{array}{ccc|ccc}
                        1 & 0 & 0 & -2 & 2 & 1 \\
                        0 & 1 & 0 & 0 & -1 & 1 \\
                        0 & 0 & 1 & 1 & 0 & -1
                    \end{array}
                \right)
                =
                \left(
                    \begin{array}{c|c}
                        E_n & B
                    \end{array}
                \right) $ \\\\
                $ \therefore
                A^{-1} =
                \left(
                    \begin{array}{ccc}
                        -2 & 2 & 1 \\
                        0 & -1 & 1 \\
                        1 & 0 & -1
                    \end{array}
                \right) $

            \item[(2)]
                $ A =
                \left(
                    \begin{array}{cccc}
                        1 & -1 & 0 & 0 \\
                        0 & 1 & -1 & 0 \\
                        0 & 0 & 1 & -1 \\
                        0 & 0 & 0 & 1
                    \end{array}
                \right) $ \\\\
                $ \left(
                    \begin{array}{c|c}
                        A & E_n
                    \end{array}
                \right)
                =
                \left(
                    \begin{array}{cccc|cccc}
                        1 & -1 & 0 & 0 & 1 & 0 & 0 & 0 \\
                        0 & 1 & -1 & 0 & 0 & 1 & 0 & 0 \\
                        0 & 0 & 1 & -1 & 0 & 0 & 1 & 0 \\
                        0 & 0 & 0 & 1 & 0 & 0 & 0 & 1
                    \end{array}
                \right) $ \\
                $ \longleftrightarrow
                \left(
                    \begin{array}{cccc|cccc}
                        1 & 0 & -1 & 0 & 1 & 1 & 0 & 0 \\
                        0 & 1 & -1 & 0 & 0 & 1 & 0 & 0 \\
                        0 & 0 & 1 & -1 & 0 & 0 & 1 & 0 \\
                        0 & 0 & 0 & 1 & 0 & 0 & 0 & 1
                    \end{array}
                \right)
                \longleftrightarrow
                \left(
                    \begin{array}{cccc|cccc}
                        1 & 0 & 0 & -1 & 1 & 1 & 1 & 0 \\
                        0 & 1 & 0 & -1 & 0 & 1 & 1 & 0 \\
                        0 & 0 & 1 & -1 & 0 & 0 & 1 & 0 \\
                        0 & 0 & 0 & 1 & 0 & 0 & 0 & 1
                    \end{array}
                \right) \\
                \longleftrightarrow
                \left(
                    \begin{array}{cccc|cccc}
                        1 & 0 & 0 & 0 & 1 & 1 & 1 & 1 \\
                        0 & 1 & 0 & 0 & 0 & 1 & 1 & 1 \\
                        0 & 0 & 1 & 0 & 0 & 0 & 1 & 1 \\
                        0 & 0 & 0 & 1 & 0 & 0 & 0 & 1
                    \end{array}
                \right)
                =
                \left(
                    \begin{array}{c|c}
                        E_n & B
                    \end{array}
                \right) $ \\
                $ \therefore
                A^{-1} =
                \left(
                    \begin{array}{cccc}
                        1 & 1 & 1 & 1 \\
                        0 & 1 & 1 & 1 \\
                        0 & 0 & 1 & 1 \\
                        0 & 0 & 0 & 1
                    \end{array}
                \right) $

        \end{spacing}
    \end{description}

\end{document}