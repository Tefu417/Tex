\documentclass[fleqn]{jsarticle}
\setlength{\headsep}{15truemm}

\usepackage{amsmath, amssymb}
\usepackage{setspace}
\usepackage{nccmath}
\usepackage{type1cm}
\usepackage{multicol}

\title{基礎数理演習課題11}
\author{21716070 縫嶋慧深}

\begin{document}

	\maketitle

    \vspace{-5mm} \section*{1}
    次の不定積分を求めて下さい。(積分定数として$C, C_1, C_2, \cdots $を断らずに用いてよい)

    \begin{description}
        \setlength{\itemsep}{0.5cm}

        \begin{multicols}{2}

            \item[(1)]
                \begin{flalign*}
                    & \hspace*{-10mm} \int\left(\cfrac{e}{x} - \cfrac{x}{\sqrt{1-x^2}}\right)dx \\
                    & \hspace*{2mm} = elog{x} + \sqrt{1-x^2} + C
                \end{flalign*}


            \item[(2)]
                \begin{flalign*}
                    & \hspace*{-10mm} \int\left(x^3 + \sqrt[3]{x} + \cfrac{1}{\sqrt[3]{x}}\right)dx \\
                    & \hspace*{2mm} = \cfrac{1}{4}\left(3x^{\frac{4}{3}} + 6x^{\frac{2}{3}} + x^4\right) + C_1
                \end{flalign*}

        \end{multicols}

        \begin{multicols}{2}

            \item[(3)]
                \begin{flalign*}
                    & \hspace*{-10mm} \int\cfrac{1}{4+x^2}dx \\
                    & \hspace*{2mm} = \cfrac{1}{2}\tan^{-1}{\cfrac{x}{2}} + C_2
                \end{flalign*}

            \item[(4)]
                \begin{flalign*}
                    & \hspace*{-10mm} \int\cfrac{2}{\sqrt{2-x^2}}dx \\
                    & \hspace*{2mm} = 2\sin^{-1}{\cfrac{x}{\sqrt{2}}} + C_3
                \end{flalign*}

        \end{multicols}

        \begin{multicols}{2}

            \item[(5)]
                \begin{flalign*}
                    & \hspace*{-10mm} \int x\sin{x}dx \\
                    & \hspace*{2mm} = \sin{x} - x\cos{x} + C_4
                \end{flalign*}

            \item[(6)]
                \begin{flalign*}
                    & \hspace*{-10mm} \int\sin{4x}dx \\
                    & \hspace*{2mm} = -\cfrac{1}{4}\cos{4x} + C_5
                \end{flalign*}

        \end{multicols}

        \begin{multicols}{2}

            \item[(7)]
                \begin{flalign*}
                    & \hspace*{-10mm} \int xlog{x}dx \\
                    & \hspace*{2mm} = \cfrac{1}{4}x^2(2log{x}- 1) + C_6
                \end{flalign*}

            \item[(8)]
                \begin{flalign*}
                    & \hspace*{-10mm} \int x\cos{x^2}dx \\
                    & \hspace*{2mm} = \cfrac{\sin{x^2}}{2} + C_7
                \end{flalign*}

        \end{multicols}

        \begin{multicols}{2}

            \item[(9)]
                \begin{flalign*}
                    & \hspace*{-10mm} \int x^2e^x dx \\
                    & \hspace*{2mm} = e^x(x^2 - 2x + 2) + C_8
                \end{flalign*}

            \item[(10)]
                \begin{flalign*}
                    & \hspace*{-10mm} \int x^2\sqrt{x^3 + 1}dx \\
                    & \hspace*{2mm} = \cfrac{2}{9}\left(x^3 + 1\right)^{\frac{3}{2}} + C_9
                \end{flalign*}

        \end{multicols}

    \end{description}

    \newpage

    \section*{2}
    次の定積分を求めて下さい。

    \begin{description}
        \setlength{\itemsep}{0.5cm}

        \begin{multicols}{2}

            \item[(1)]
                \begin{flalign*}
                    & \hspace*{-10mm} \int_1^3 x^2log{x} dx \\
                    & \hspace*{-2mm} = \left[\cfrac{1}{3}x^3log{x}- \cfrac{x^3}{9}\right]_1^3 \\
                    & \hspace*{-2mm} = \left(9log{3} - 3 \right) - \left(-\cfrac{1}{9}\right) \\
                    & \hspace*{-2mm} = 9log{3} - \cfrac{26}{9}
                \end{flalign*}

            \item[(2)]
                \begin{flalign*}
                    & \hspace*{-10mm} \int_2^3 xe^{x^2} dx \\
                    & \hspace*{-2mm} = \left[\cfrac{e^{x^2}}{2}\right]_2^3 \\
                    & \hspace*{-2mm} = \cfrac{e^9}{2} - \cfrac{e^4}{2} \\
                    & \hspace*{-2mm} = \cfrac{1}{2}e^4(e^5 - 1)
                \end{flalign*}

        \end{multicols}

    \end{description}

    \section*{3}
    $x = 0$ と $x = 3$ の間でグラフ $y = x^3 - 4x$ と $x軸$ に挟まれた領域の(通常の)面積を求めて下さい。

    \begin{flalign*}
        & \hspace*{-4mm} \int x^3 - 4x dx = \cfrac{x^4}{4} - 2x^2 + C \\
        & \hspace*{-4mm} \cfrac{x^4}{4} - 2x^2 = 0 \ \Leftrightarrow \ x = 0, 2\sqrt{2}, -2\sqrt{2} \\
        & \hspace*{-4mm} S = \int_0^3 x^3 - 4x dx = -\int_0^{2\sqrt{2}} x^3 - 4x dx + \int_{2\sqrt{2}}^3 x^3 - 4x dx \\
        & \hspace*{25mm} = -\left[\cfrac{x^4}{4} - 2x^2\right]_0^{2\sqrt{2}} + \left[\cfrac{x^4}{4} - 2x^2\right]_{2\sqrt{2}}^3 \\
        & \hspace*{25mm} = -(16 - 16) + \left\{\left(\cfrac{81}{4} - 18\right) - 0 \right\} \\
        & \hspace*{25mm} = \cfrac{9}{4} \\
    \end{flalign*}

\end{document}