\documentclass{jsarticle}

\usepackage{amsmath}
\usepackage{setspace}

\title{基礎数理演習課題2}
\author{21716070 縫嶋慧深}

\begin{document}

	\maketitle

	\section*{0}
	次の行列の計算をして下さい。

	\begin{description}
		\setlength{\itemsep}{0.5cm}

		\item[(1)]
			$ \begin{pmatrix}
				-2 & 6 \\
				5 & 4 \\
				1 & -3
			\end{pmatrix} $
			-
			$ 3 \begin{pmatrix}
				-2 & 3 \\
				4 & -1 \\
				0 & 2
			\end{pmatrix} $
			=
			$ \begin{pmatrix}
				4 & -3 \\
				-7 & 7 \\
				1 & -9
			\end{pmatrix} $

		\item[(2)]
			$ \begin{pmatrix}
				-1 & 2 \\
				2 & 1 \\
				1 & -1
			\end{pmatrix} $
			$ \begin{pmatrix}
				4 & 0 \\
				-3 & 1
			\end{pmatrix} $
			=
			$ \begin{pmatrix}
				-10 & 2 \\
				5 & 1 \\
				7 & -1
			\end{pmatrix} $

		\item[(3)]
			$ \begin{pmatrix}
				6 & 1 \\
				0 & -5
			\end{pmatrix} $
			$ \begin{pmatrix}
				8 & -1 & 5 \\
				-7 & 3 & 0
			\end{pmatrix} $
			=
			$ \begin{pmatrix}
				41 & -3 & 30 \\
				35 & -15 & 0
			\end{pmatrix} $

		\item[(4)]
			$ \begin{pmatrix}
				-1 \\
				-2 \\
				2
			\end{pmatrix} $
			$ \begin{pmatrix}
				1 & 0 & 2 & -5
			\end{pmatrix} $
			=
			$ \begin{pmatrix}
				-1 & 0 & -2 & 5 \\
				-2 & 0 & -4 & 10 \\
				2 & 0 & 4 & -10
			\end{pmatrix} $

		\item[(5)]
			$ \begin{pmatrix}
				1 & 2 & 3 \\
				-1 & -1 & -2 \\
				3 & -2 & 1
			\end{pmatrix} $
			$ \begin{pmatrix}
				1 & 0 & -1 \\
				1 & 0 & -1 \\
				-1 & 0 & 1
			\end{pmatrix} $
			=
			$ \begin{pmatrix}
				0 & 0 & 0 \\
				0 & 0 & 0 \\
				0 & 0 & 0
			\end{pmatrix} $

		\item[(6)]
			$ \begin{pmatrix}
				2 & -9 \\
				4 & 6 \\
				3 & 0
			\end{pmatrix} $
			$ \begin{pmatrix}
				1 & 0 & 0 \\
				0 & 0 & 1 \\
				0 & 1 & 0
			\end{pmatrix} $

			行列の次元が矛盾しているため、計算できない。
	\end{description}

	\newpage

	\section{}
	次の行列が逆行列を持つか判定し、持つならば逆行列も求めて下さい。

	\begin{description}

		\setlength{\itemsep}{0.5cm}
		\begin{spacing}{1.2}

		\item[(1)]
			$A$ =
			$ \begin{pmatrix}
				2 & 1 \\
				5 & 3
			\end{pmatrix} $

			$ det(A) = 6 - 5 = 1 \neq 0 $

			$ A^{-1} = \frac{1}{1} $
			$ \begin{pmatrix}
				3 & -1 \\
				-5 & 2
			\end{pmatrix} $
			=
			$ \begin{pmatrix}
				3 & -1 \\
				-5 & 2
			\end{pmatrix} $

		\item[(2)]
			$A$ =
			$ \begin{pmatrix}
				-1 & 2 \\
				3 & -4
			\end{pmatrix} $

			$ det(A) = 4 - 6 = -2 \neq 0 $

			$ A^{-1} = -\frac{1}{2} $
			$ \begin{pmatrix}
				4 & 2 \\
				3 & 1
			\end{pmatrix} $
			=
			$ \begin{pmatrix}
				2 & 1 \\
				\frac{3}{2} & \frac{1}{2}
			\end{pmatrix} $

		\item[(3)]
			$A$ =
			$ \begin{pmatrix}
				5 & 6 \\
				6 & 7
			\end{pmatrix} $

			$ det(A) = 35 - 36 = -1 \neq 0 $

			$A^{-1}$ =
			$ -\begin{pmatrix}
				7 & -6 \\
				-6 & 5
			\end{pmatrix} $
			=
			$ \begin{pmatrix}
				-7 & 6 \\
				6 & -5
			\end{pmatrix} $

		\item[(4)]
			$ \begin{pmatrix}
				3 & -1 & 4 \\
				4 & 1 & 3 \\
				1 & 3 & -2
			\end{pmatrix} $

			$ det(A) = -6 + (-3) + 48 - 4 - 8 - 27 = 0 $

			$ det(A) = 0 $ より、逆行列を持たない。

		\item[(5)]
			$A$ =
			$ \begin{pmatrix}
				1 & 1 & 2 \\
				2 & 4 & 4 \\
				3 & 3 & 6
			\end{pmatrix} $

			$ det(A) = 24 + 12 + 12 - 24 - 12 - 12 = 0 $

			$ det(A) = 0 $ より、逆行列を持たない。

		\item[(6)]
			$A$ =
			$ \begin{pmatrix}
				-3
			\end{pmatrix} $

			行列 $A$ は正則行列でないため、逆行列を持たない。

		\end{spacing}
	\end{description}
\end{document}